\documentclass[11pt]{article}

    \usepackage[breakable]{tcolorbox}
    \usepackage{parskip} % Stop auto-indenting (to mimic markdown behaviour)
    

    % Basic figure setup, for now with no caption control since it's done
    % automatically by Pandoc (which extracts ![](path) syntax from Markdown).
    \usepackage{graphicx}
    % Maintain compatibility with old templates. Remove in nbconvert 6.0
    \let\Oldincludegraphics\includegraphics
    % Ensure that by default, figures have no caption (until we provide a
    % proper Figure object with a Caption API and a way to capture that
    % in the conversion process - todo).
    \usepackage{caption}
    \DeclareCaptionFormat{nocaption}{}
    \captionsetup{format=nocaption,aboveskip=0pt,belowskip=0pt}

    \usepackage{float}
    \floatplacement{figure}{H} % forces figures to be placed at the correct location
    \usepackage{xcolor} % Allow colors to be defined
    \usepackage{enumerate} % Needed for markdown enumerations to work
    \usepackage{geometry} % Used to adjust the document margins
    \usepackage{amsmath} % Equations
    \usepackage{amssymb} % Equations
    \usepackage{textcomp} % defines textquotesingle
    % Hack from http://tex.stackexchange.com/a/47451/13684:
    \AtBeginDocument{%
        \def\PYZsq{\textquotesingle}% Upright quotes in Pygmentized code
    }
    \usepackage{upquote} % Upright quotes for verbatim code
    \usepackage{eurosym} % defines \euro

    \usepackage{iftex}
    \ifPDFTeX
        \usepackage[T1]{fontenc}
        \IfFileExists{alphabeta.sty}{
              \usepackage{alphabeta}
          }{
              \usepackage[mathletters]{ucs}
              \usepackage[utf8x]{inputenc}
          }
    \else
        \usepackage{fontspec}
        \usepackage{unicode-math}
    \fi

    \usepackage{fancyvrb} % verbatim replacement that allows latex
    \usepackage{grffile} % extends the file name processing of package graphics 
                         % to support a larger range
    \makeatletter % fix for old versions of grffile with XeLaTeX
    \@ifpackagelater{grffile}{2019/11/01}
    {
      % Do nothing on new versions
    }
    {
      \def\Gread@@xetex#1{%
        \IfFileExists{"\Gin@base".bb}%
        {\Gread@eps{\Gin@base.bb}}%
        {\Gread@@xetex@aux#1}%
      }
    }
    \makeatother
    \usepackage[Export]{adjustbox} % Used to constrain images to a maximum size
    \adjustboxset{max size={0.9\linewidth}{0.9\paperheight}}

    % The hyperref package gives us a pdf with properly built
    % internal navigation ('pdf bookmarks' for the table of contents,
    % internal cross-reference links, web links for URLs, etc.)
    \usepackage{hyperref}
    % The default LaTeX title has an obnoxious amount of whitespace. By default,
    % titling removes some of it. It also provides customization options.
    \usepackage{titling}
    \usepackage{longtable} % longtable support required by pandoc >1.10
    \usepackage{booktabs}  % table support for pandoc > 1.12.2
    \usepackage{array}     % table support for pandoc >= 2.11.3
    \usepackage{calc}      % table minipage width calculation for pandoc >= 2.11.1
    \usepackage[inline]{enumitem} % IRkernel/repr support (it uses the enumerate* environment)
    \usepackage[normalem]{ulem} % ulem is needed to support strikethroughs (\sout)
                                % normalem makes italics be italics, not underlines
    \usepackage{mathrsfs}
    

    
    % Colors for the hyperref package
    \definecolor{urlcolor}{rgb}{0,.145,.698}
    \definecolor{linkcolor}{rgb}{.71,0.21,0.01}
    \definecolor{citecolor}{rgb}{.12,.54,.11}

    % ANSI colors
    \definecolor{ansi-black}{HTML}{3E424D}
    \definecolor{ansi-black-intense}{HTML}{282C36}
    \definecolor{ansi-red}{HTML}{E75C58}
    \definecolor{ansi-red-intense}{HTML}{B22B31}
    \definecolor{ansi-green}{HTML}{00A250}
    \definecolor{ansi-green-intense}{HTML}{007427}
    \definecolor{ansi-yellow}{HTML}{DDB62B}
    \definecolor{ansi-yellow-intense}{HTML}{B27D12}
    \definecolor{ansi-blue}{HTML}{208FFB}
    \definecolor{ansi-blue-intense}{HTML}{0065CA}
    \definecolor{ansi-magenta}{HTML}{D160C4}
    \definecolor{ansi-magenta-intense}{HTML}{A03196}
    \definecolor{ansi-cyan}{HTML}{60C6C8}
    \definecolor{ansi-cyan-intense}{HTML}{258F8F}
    \definecolor{ansi-white}{HTML}{C5C1B4}
    \definecolor{ansi-white-intense}{HTML}{A1A6B2}
    \definecolor{ansi-default-inverse-fg}{HTML}{FFFFFF}
    \definecolor{ansi-default-inverse-bg}{HTML}{000000}

    % common color for the border for error outputs.
    \definecolor{outerrorbackground}{HTML}{FFDFDF}

    % commands and environments needed by pandoc snippets
    % extracted from the output of `pandoc -s`
    \providecommand{\tightlist}{%
      \setlength{\itemsep}{0pt}\setlength{\parskip}{0pt}}
    \DefineVerbatimEnvironment{Highlighting}{Verbatim}{commandchars=\\\{\}}
    % Add ',fontsize=\small' for more characters per line
    \newenvironment{Shaded}{}{}
    \newcommand{\KeywordTok}[1]{\textcolor[rgb]{0.00,0.44,0.13}{\textbf{{#1}}}}
    \newcommand{\DataTypeTok}[1]{\textcolor[rgb]{0.56,0.13,0.00}{{#1}}}
    \newcommand{\DecValTok}[1]{\textcolor[rgb]{0.25,0.63,0.44}{{#1}}}
    \newcommand{\BaseNTok}[1]{\textcolor[rgb]{0.25,0.63,0.44}{{#1}}}
    \newcommand{\FloatTok}[1]{\textcolor[rgb]{0.25,0.63,0.44}{{#1}}}
    \newcommand{\CharTok}[1]{\textcolor[rgb]{0.25,0.44,0.63}{{#1}}}
    \newcommand{\StringTok}[1]{\textcolor[rgb]{0.25,0.44,0.63}{{#1}}}
    \newcommand{\CommentTok}[1]{\textcolor[rgb]{0.38,0.63,0.69}{\textit{{#1}}}}
    \newcommand{\OtherTok}[1]{\textcolor[rgb]{0.00,0.44,0.13}{{#1}}}
    \newcommand{\AlertTok}[1]{\textcolor[rgb]{1.00,0.00,0.00}{\textbf{{#1}}}}
    \newcommand{\FunctionTok}[1]{\textcolor[rgb]{0.02,0.16,0.49}{{#1}}}
    \newcommand{\RegionMarkerTok}[1]{{#1}}
    \newcommand{\ErrorTok}[1]{\textcolor[rgb]{1.00,0.00,0.00}{\textbf{{#1}}}}
    \newcommand{\NormalTok}[1]{{#1}}
    
    % Additional commands for more recent versions of Pandoc
    \newcommand{\ConstantTok}[1]{\textcolor[rgb]{0.53,0.00,0.00}{{#1}}}
    \newcommand{\SpecialCharTok}[1]{\textcolor[rgb]{0.25,0.44,0.63}{{#1}}}
    \newcommand{\VerbatimStringTok}[1]{\textcolor[rgb]{0.25,0.44,0.63}{{#1}}}
    \newcommand{\SpecialStringTok}[1]{\textcolor[rgb]{0.73,0.40,0.53}{{#1}}}
    \newcommand{\ImportTok}[1]{{#1}}
    \newcommand{\DocumentationTok}[1]{\textcolor[rgb]{0.73,0.13,0.13}{\textit{{#1}}}}
    \newcommand{\AnnotationTok}[1]{\textcolor[rgb]{0.38,0.63,0.69}{\textbf{\textit{{#1}}}}}
    \newcommand{\CommentVarTok}[1]{\textcolor[rgb]{0.38,0.63,0.69}{\textbf{\textit{{#1}}}}}
    \newcommand{\VariableTok}[1]{\textcolor[rgb]{0.10,0.09,0.49}{{#1}}}
    \newcommand{\ControlFlowTok}[1]{\textcolor[rgb]{0.00,0.44,0.13}{\textbf{{#1}}}}
    \newcommand{\OperatorTok}[1]{\textcolor[rgb]{0.40,0.40,0.40}{{#1}}}
    \newcommand{\BuiltInTok}[1]{{#1}}
    \newcommand{\ExtensionTok}[1]{{#1}}
    \newcommand{\PreprocessorTok}[1]{\textcolor[rgb]{0.74,0.48,0.00}{{#1}}}
    \newcommand{\AttributeTok}[1]{\textcolor[rgb]{0.49,0.56,0.16}{{#1}}}
    \newcommand{\InformationTok}[1]{\textcolor[rgb]{0.38,0.63,0.69}{\textbf{\textit{{#1}}}}}
    \newcommand{\WarningTok}[1]{\textcolor[rgb]{0.38,0.63,0.69}{\textbf{\textit{{#1}}}}}
    
    
    % Define a nice break command that doesn't care if a line doesn't already
    % exist.
    \def\br{\hspace*{\fill} \\* }
    % Math Jax compatibility definitions
    \def\gt{>}
    \def\lt{<}
    \let\Oldtex\TeX
    \let\Oldlatex\LaTeX
    \renewcommand{\TeX}{\textrm{\Oldtex}}
    \renewcommand{\LaTeX}{\textrm{\Oldlatex}}
    % Document parameters
    % Document title
    \title{Q4\_Jump\_Diffusion\_Implied\_Volatility\_Final}
    
    
    
    
    
% Pygments definitions
\makeatletter
\def\PY@reset{\let\PY@it=\relax \let\PY@bf=\relax%
    \let\PY@ul=\relax \let\PY@tc=\relax%
    \let\PY@bc=\relax \let\PY@ff=\relax}
\def\PY@tok#1{\csname PY@tok@#1\endcsname}
\def\PY@toks#1+{\ifx\relax#1\empty\else%
    \PY@tok{#1}\expandafter\PY@toks\fi}
\def\PY@do#1{\PY@bc{\PY@tc{\PY@ul{%
    \PY@it{\PY@bf{\PY@ff{#1}}}}}}}
\def\PY#1#2{\PY@reset\PY@toks#1+\relax+\PY@do{#2}}

\@namedef{PY@tok@w}{\def\PY@tc##1{\textcolor[rgb]{0.73,0.73,0.73}{##1}}}
\@namedef{PY@tok@c}{\let\PY@it=\textit\def\PY@tc##1{\textcolor[rgb]{0.24,0.48,0.48}{##1}}}
\@namedef{PY@tok@cp}{\def\PY@tc##1{\textcolor[rgb]{0.61,0.40,0.00}{##1}}}
\@namedef{PY@tok@k}{\let\PY@bf=\textbf\def\PY@tc##1{\textcolor[rgb]{0.00,0.50,0.00}{##1}}}
\@namedef{PY@tok@kp}{\def\PY@tc##1{\textcolor[rgb]{0.00,0.50,0.00}{##1}}}
\@namedef{PY@tok@kt}{\def\PY@tc##1{\textcolor[rgb]{0.69,0.00,0.25}{##1}}}
\@namedef{PY@tok@o}{\def\PY@tc##1{\textcolor[rgb]{0.40,0.40,0.40}{##1}}}
\@namedef{PY@tok@ow}{\let\PY@bf=\textbf\def\PY@tc##1{\textcolor[rgb]{0.67,0.13,1.00}{##1}}}
\@namedef{PY@tok@nb}{\def\PY@tc##1{\textcolor[rgb]{0.00,0.50,0.00}{##1}}}
\@namedef{PY@tok@nf}{\def\PY@tc##1{\textcolor[rgb]{0.00,0.00,1.00}{##1}}}
\@namedef{PY@tok@nc}{\let\PY@bf=\textbf\def\PY@tc##1{\textcolor[rgb]{0.00,0.00,1.00}{##1}}}
\@namedef{PY@tok@nn}{\let\PY@bf=\textbf\def\PY@tc##1{\textcolor[rgb]{0.00,0.00,1.00}{##1}}}
\@namedef{PY@tok@ne}{\let\PY@bf=\textbf\def\PY@tc##1{\textcolor[rgb]{0.80,0.25,0.22}{##1}}}
\@namedef{PY@tok@nv}{\def\PY@tc##1{\textcolor[rgb]{0.10,0.09,0.49}{##1}}}
\@namedef{PY@tok@no}{\def\PY@tc##1{\textcolor[rgb]{0.53,0.00,0.00}{##1}}}
\@namedef{PY@tok@nl}{\def\PY@tc##1{\textcolor[rgb]{0.46,0.46,0.00}{##1}}}
\@namedef{PY@tok@ni}{\let\PY@bf=\textbf\def\PY@tc##1{\textcolor[rgb]{0.44,0.44,0.44}{##1}}}
\@namedef{PY@tok@na}{\def\PY@tc##1{\textcolor[rgb]{0.41,0.47,0.13}{##1}}}
\@namedef{PY@tok@nt}{\let\PY@bf=\textbf\def\PY@tc##1{\textcolor[rgb]{0.00,0.50,0.00}{##1}}}
\@namedef{PY@tok@nd}{\def\PY@tc##1{\textcolor[rgb]{0.67,0.13,1.00}{##1}}}
\@namedef{PY@tok@s}{\def\PY@tc##1{\textcolor[rgb]{0.73,0.13,0.13}{##1}}}
\@namedef{PY@tok@sd}{\let\PY@it=\textit\def\PY@tc##1{\textcolor[rgb]{0.73,0.13,0.13}{##1}}}
\@namedef{PY@tok@si}{\let\PY@bf=\textbf\def\PY@tc##1{\textcolor[rgb]{0.64,0.35,0.47}{##1}}}
\@namedef{PY@tok@se}{\let\PY@bf=\textbf\def\PY@tc##1{\textcolor[rgb]{0.67,0.36,0.12}{##1}}}
\@namedef{PY@tok@sr}{\def\PY@tc##1{\textcolor[rgb]{0.64,0.35,0.47}{##1}}}
\@namedef{PY@tok@ss}{\def\PY@tc##1{\textcolor[rgb]{0.10,0.09,0.49}{##1}}}
\@namedef{PY@tok@sx}{\def\PY@tc##1{\textcolor[rgb]{0.00,0.50,0.00}{##1}}}
\@namedef{PY@tok@m}{\def\PY@tc##1{\textcolor[rgb]{0.40,0.40,0.40}{##1}}}
\@namedef{PY@tok@gh}{\let\PY@bf=\textbf\def\PY@tc##1{\textcolor[rgb]{0.00,0.00,0.50}{##1}}}
\@namedef{PY@tok@gu}{\let\PY@bf=\textbf\def\PY@tc##1{\textcolor[rgb]{0.50,0.00,0.50}{##1}}}
\@namedef{PY@tok@gd}{\def\PY@tc##1{\textcolor[rgb]{0.63,0.00,0.00}{##1}}}
\@namedef{PY@tok@gi}{\def\PY@tc##1{\textcolor[rgb]{0.00,0.52,0.00}{##1}}}
\@namedef{PY@tok@gr}{\def\PY@tc##1{\textcolor[rgb]{0.89,0.00,0.00}{##1}}}
\@namedef{PY@tok@ge}{\let\PY@it=\textit}
\@namedef{PY@tok@gs}{\let\PY@bf=\textbf}
\@namedef{PY@tok@gp}{\let\PY@bf=\textbf\def\PY@tc##1{\textcolor[rgb]{0.00,0.00,0.50}{##1}}}
\@namedef{PY@tok@go}{\def\PY@tc##1{\textcolor[rgb]{0.44,0.44,0.44}{##1}}}
\@namedef{PY@tok@gt}{\def\PY@tc##1{\textcolor[rgb]{0.00,0.27,0.87}{##1}}}
\@namedef{PY@tok@err}{\def\PY@bc##1{{\setlength{\fboxsep}{\string -\fboxrule}\fcolorbox[rgb]{1.00,0.00,0.00}{1,1,1}{\strut ##1}}}}
\@namedef{PY@tok@kc}{\let\PY@bf=\textbf\def\PY@tc##1{\textcolor[rgb]{0.00,0.50,0.00}{##1}}}
\@namedef{PY@tok@kd}{\let\PY@bf=\textbf\def\PY@tc##1{\textcolor[rgb]{0.00,0.50,0.00}{##1}}}
\@namedef{PY@tok@kn}{\let\PY@bf=\textbf\def\PY@tc##1{\textcolor[rgb]{0.00,0.50,0.00}{##1}}}
\@namedef{PY@tok@kr}{\let\PY@bf=\textbf\def\PY@tc##1{\textcolor[rgb]{0.00,0.50,0.00}{##1}}}
\@namedef{PY@tok@bp}{\def\PY@tc##1{\textcolor[rgb]{0.00,0.50,0.00}{##1}}}
\@namedef{PY@tok@fm}{\def\PY@tc##1{\textcolor[rgb]{0.00,0.00,1.00}{##1}}}
\@namedef{PY@tok@vc}{\def\PY@tc##1{\textcolor[rgb]{0.10,0.09,0.49}{##1}}}
\@namedef{PY@tok@vg}{\def\PY@tc##1{\textcolor[rgb]{0.10,0.09,0.49}{##1}}}
\@namedef{PY@tok@vi}{\def\PY@tc##1{\textcolor[rgb]{0.10,0.09,0.49}{##1}}}
\@namedef{PY@tok@vm}{\def\PY@tc##1{\textcolor[rgb]{0.10,0.09,0.49}{##1}}}
\@namedef{PY@tok@sa}{\def\PY@tc##1{\textcolor[rgb]{0.73,0.13,0.13}{##1}}}
\@namedef{PY@tok@sb}{\def\PY@tc##1{\textcolor[rgb]{0.73,0.13,0.13}{##1}}}
\@namedef{PY@tok@sc}{\def\PY@tc##1{\textcolor[rgb]{0.73,0.13,0.13}{##1}}}
\@namedef{PY@tok@dl}{\def\PY@tc##1{\textcolor[rgb]{0.73,0.13,0.13}{##1}}}
\@namedef{PY@tok@s2}{\def\PY@tc##1{\textcolor[rgb]{0.73,0.13,0.13}{##1}}}
\@namedef{PY@tok@sh}{\def\PY@tc##1{\textcolor[rgb]{0.73,0.13,0.13}{##1}}}
\@namedef{PY@tok@s1}{\def\PY@tc##1{\textcolor[rgb]{0.73,0.13,0.13}{##1}}}
\@namedef{PY@tok@mb}{\def\PY@tc##1{\textcolor[rgb]{0.40,0.40,0.40}{##1}}}
\@namedef{PY@tok@mf}{\def\PY@tc##1{\textcolor[rgb]{0.40,0.40,0.40}{##1}}}
\@namedef{PY@tok@mh}{\def\PY@tc##1{\textcolor[rgb]{0.40,0.40,0.40}{##1}}}
\@namedef{PY@tok@mi}{\def\PY@tc##1{\textcolor[rgb]{0.40,0.40,0.40}{##1}}}
\@namedef{PY@tok@il}{\def\PY@tc##1{\textcolor[rgb]{0.40,0.40,0.40}{##1}}}
\@namedef{PY@tok@mo}{\def\PY@tc##1{\textcolor[rgb]{0.40,0.40,0.40}{##1}}}
\@namedef{PY@tok@ch}{\let\PY@it=\textit\def\PY@tc##1{\textcolor[rgb]{0.24,0.48,0.48}{##1}}}
\@namedef{PY@tok@cm}{\let\PY@it=\textit\def\PY@tc##1{\textcolor[rgb]{0.24,0.48,0.48}{##1}}}
\@namedef{PY@tok@cpf}{\let\PY@it=\textit\def\PY@tc##1{\textcolor[rgb]{0.24,0.48,0.48}{##1}}}
\@namedef{PY@tok@c1}{\let\PY@it=\textit\def\PY@tc##1{\textcolor[rgb]{0.24,0.48,0.48}{##1}}}
\@namedef{PY@tok@cs}{\let\PY@it=\textit\def\PY@tc##1{\textcolor[rgb]{0.24,0.48,0.48}{##1}}}

\def\PYZbs{\char`\\}
\def\PYZus{\char`\_}
\def\PYZob{\char`\{}
\def\PYZcb{\char`\}}
\def\PYZca{\char`\^}
\def\PYZam{\char`\&}
\def\PYZlt{\char`\<}
\def\PYZgt{\char`\>}
\def\PYZsh{\char`\#}
\def\PYZpc{\char`\%}
\def\PYZdl{\char`\$}
\def\PYZhy{\char`\-}
\def\PYZsq{\char`\'}
\def\PYZdq{\char`\"}
\def\PYZti{\char`\~}
% for compatibility with earlier versions
\def\PYZat{@}
\def\PYZlb{[}
\def\PYZrb{]}
\makeatother


    % For linebreaks inside Verbatim environment from package fancyvrb. 
    \makeatletter
        \newbox\Wrappedcontinuationbox 
        \newbox\Wrappedvisiblespacebox 
        \newcommand*\Wrappedvisiblespace {\textcolor{red}{\textvisiblespace}} 
        \newcommand*\Wrappedcontinuationsymbol {\textcolor{red}{\llap{\tiny$\m@th\hookrightarrow$}}} 
        \newcommand*\Wrappedcontinuationindent {3ex } 
        \newcommand*\Wrappedafterbreak {\kern\Wrappedcontinuationindent\copy\Wrappedcontinuationbox} 
        % Take advantage of the already applied Pygments mark-up to insert 
        % potential linebreaks for TeX processing. 
        %        {, <, #, %, $, ' and ": go to next line. 
        %        _, }, ^, &, >, - and ~: stay at end of broken line. 
        % Use of \textquotesingle for straight quote. 
        \newcommand*\Wrappedbreaksatspecials {% 
            \def\PYGZus{\discretionary{\char`\_}{\Wrappedafterbreak}{\char`\_}}% 
            \def\PYGZob{\discretionary{}{\Wrappedafterbreak\char`\{}{\char`\{}}% 
            \def\PYGZcb{\discretionary{\char`\}}{\Wrappedafterbreak}{\char`\}}}% 
            \def\PYGZca{\discretionary{\char`\^}{\Wrappedafterbreak}{\char`\^}}% 
            \def\PYGZam{\discretionary{\char`\&}{\Wrappedafterbreak}{\char`\&}}% 
            \def\PYGZlt{\discretionary{}{\Wrappedafterbreak\char`\<}{\char`\<}}% 
            \def\PYGZgt{\discretionary{\char`\>}{\Wrappedafterbreak}{\char`\>}}% 
            \def\PYGZsh{\discretionary{}{\Wrappedafterbreak\char`\#}{\char`\#}}% 
            \def\PYGZpc{\discretionary{}{\Wrappedafterbreak\char`\%}{\char`\%}}% 
            \def\PYGZdl{\discretionary{}{\Wrappedafterbreak\char`\$}{\char`\$}}% 
            \def\PYGZhy{\discretionary{\char`\-}{\Wrappedafterbreak}{\char`\-}}% 
            \def\PYGZsq{\discretionary{}{\Wrappedafterbreak\textquotesingle}{\textquotesingle}}% 
            \def\PYGZdq{\discretionary{}{\Wrappedafterbreak\char`\"}{\char`\"}}% 
            \def\PYGZti{\discretionary{\char`\~}{\Wrappedafterbreak}{\char`\~}}% 
        } 
        % Some characters . , ; ? ! / are not pygmentized. 
        % This macro makes them "active" and they will insert potential linebreaks 
        \newcommand*\Wrappedbreaksatpunct {% 
            \lccode`\~`\.\lowercase{\def~}{\discretionary{\hbox{\char`\.}}{\Wrappedafterbreak}{\hbox{\char`\.}}}% 
            \lccode`\~`\,\lowercase{\def~}{\discretionary{\hbox{\char`\,}}{\Wrappedafterbreak}{\hbox{\char`\,}}}% 
            \lccode`\~`\;\lowercase{\def~}{\discretionary{\hbox{\char`\;}}{\Wrappedafterbreak}{\hbox{\char`\;}}}% 
            \lccode`\~`\:\lowercase{\def~}{\discretionary{\hbox{\char`\:}}{\Wrappedafterbreak}{\hbox{\char`\:}}}% 
            \lccode`\~`\?\lowercase{\def~}{\discretionary{\hbox{\char`\?}}{\Wrappedafterbreak}{\hbox{\char`\?}}}% 
            \lccode`\~`\!\lowercase{\def~}{\discretionary{\hbox{\char`\!}}{\Wrappedafterbreak}{\hbox{\char`\!}}}% 
            \lccode`\~`\/\lowercase{\def~}{\discretionary{\hbox{\char`\/}}{\Wrappedafterbreak}{\hbox{\char`\/}}}% 
            \catcode`\.\active
            \catcode`\,\active 
            \catcode`\;\active
            \catcode`\:\active
            \catcode`\?\active
            \catcode`\!\active
            \catcode`\/\active 
            \lccode`\~`\~ 	
        }
    \makeatother

    \let\OriginalVerbatim=\Verbatim
    \makeatletter
    \renewcommand{\Verbatim}[1][1]{%
        %\parskip\z@skip
        \sbox\Wrappedcontinuationbox {\Wrappedcontinuationsymbol}%
        \sbox\Wrappedvisiblespacebox {\FV@SetupFont\Wrappedvisiblespace}%
        \def\FancyVerbFormatLine ##1{\hsize\linewidth
            \vtop{\raggedright\hyphenpenalty\z@\exhyphenpenalty\z@
                \doublehyphendemerits\z@\finalhyphendemerits\z@
                \strut ##1\strut}%
        }%
        % If the linebreak is at a space, the latter will be displayed as visible
        % space at end of first line, and a continuation symbol starts next line.
        % Stretch/shrink are however usually zero for typewriter font.
        \def\FV@Space {%
            \nobreak\hskip\z@ plus\fontdimen3\font minus\fontdimen4\font
            \discretionary{\copy\Wrappedvisiblespacebox}{\Wrappedafterbreak}
            {\kern\fontdimen2\font}%
        }%
        
        % Allow breaks at special characters using \PYG... macros.
        \Wrappedbreaksatspecials
        % Breaks at punctuation characters . , ; ? ! and / need catcode=\active 	
        \OriginalVerbatim[#1,codes*=\Wrappedbreaksatpunct]%
    }
    \makeatother

    % Exact colors from NB
    \definecolor{incolor}{HTML}{303F9F}
    \definecolor{outcolor}{HTML}{D84315}
    \definecolor{cellborder}{HTML}{CFCFCF}
    \definecolor{cellbackground}{HTML}{F7F7F7}
    
    % prompt
    \makeatletter
    \newcommand{\boxspacing}{\kern\kvtcb@left@rule\kern\kvtcb@boxsep}
    \makeatother
    \newcommand{\prompt}[4]{
        {\ttfamily\llap{{\color{#2}[#3]:\hspace{3pt}#4}}\vspace{-\baselineskip}}
    }
    

    
    % Prevent overflowing lines due to hard-to-break entities
    \sloppy 
    % Setup hyperref package
    \hypersetup{
      breaklinks=true,  % so long urls are correctly broken across lines
      colorlinks=true,
      urlcolor=urlcolor,
      linkcolor=linkcolor,
      citecolor=citecolor,
      }
    % Slightly bigger margins than the latex defaults
    
    \geometry{verbose,tmargin=1in,bmargin=1in,lmargin=1in,rmargin=1in}
    
    

\begin{document}
    
    \maketitle
    
    

    
    \hypertarget{problem-4-jump-diffusion-implied-volatility}{%
\subsection{Problem 4: Jump Diffusion Implied
Volatility}\label{problem-4-jump-diffusion-implied-volatility}}

    \hypertarget{import-libraries}{%
\subsubsection{Import Libraries}\label{import-libraries}}

    \begin{tcolorbox}[breakable, size=fbox, boxrule=1pt, pad at break*=1mm,colback=cellbackground, colframe=cellborder]
\prompt{In}{incolor}{1}{\boxspacing}
\begin{Verbatim}[commandchars=\\\{\}]
\PY{k+kn}{import} \PY{n+nn}{numpy} \PY{k}{as} \PY{n+nn}{np}
\PY{k+kn}{from} \PY{n+nn}{scipy}\PY{n+nn}{.}\PY{n+nn}{stats} \PY{k+kn}{import} \PY{n}{norm}\PY{p}{,} \PY{n}{poisson}
\PY{k+kn}{import} \PY{n+nn}{matplotlib}\PY{n+nn}{.}\PY{n+nn}{pyplot} \PY{k}{as} \PY{n+nn}{plt}
\PY{k+kn}{from} \PY{n+nn}{scipy}\PY{n+nn}{.}\PY{n+nn}{signal} \PY{k+kn}{import} \PY{n}{savgol\PYZus{}filter}
\PY{k+kn}{from} \PY{n+nn}{time} \PY{k+kn}{import} \PY{n}{time}
\end{Verbatim}
\end{tcolorbox}

    \hypertarget{define-functions}{%
\subsubsection{Define Functions}\label{define-functions}}

    \hypertarget{function-1-implement-a-single-simulation-for-prices-using-jump-diffusion}{%
\paragraph{Function 1: Implement a single simulation for prices using
jump
diffusion}\label{function-1-implement-a-single-simulation-for-prices-using-jump-diffusion}}

    \begin{tcolorbox}[breakable, size=fbox, boxrule=1pt, pad at break*=1mm,colback=cellbackground, colframe=cellborder]
\prompt{In}{incolor}{2}{\boxspacing}
\begin{Verbatim}[commandchars=\\\{\}]
\PY{k}{def} \PY{n+nf}{jump\PYZus{}diffusion}\PY{p}{(}\PY{n}{S0}\PY{p}{,} \PY{n}{r}\PY{p}{,} \PY{n}{dt}\PY{p}{,} \PY{n}{sigma}\PY{p}{,} \PY{n}{steps}\PY{p}{,} \PY{n}{n\PYZus{}MC}\PY{p}{,} \PY{n}{lmbda}\PY{p}{,} \PY{n}{a}\PY{p}{,} \PY{n}{b}\PY{p}{)}\PY{p}{:}
    \PY{c+c1}{\PYZsh{} Generate paths for the log\PYZhy{}price and jump process}
    \PY{n}{Nt} \PY{o}{=} \PY{n}{np}\PY{o}{.}\PY{n}{random}\PY{o}{.}\PY{n}{poisson}\PY{p}{(}\PY{n}{lmbda} \PY{o}{*} \PY{n}{dt}\PY{p}{,} \PY{n}{size}\PY{o}{=}\PY{p}{(}\PY{n}{n\PYZus{}MC}\PY{p}{,} \PY{n}{steps}\PY{p}{)}\PY{p}{)}
    \PY{n}{Y} \PY{o}{=} \PY{n}{np}\PY{o}{.}\PY{n}{exp}\PY{p}{(}\PY{n}{a} \PY{o}{+} \PY{n}{b} \PY{o}{*} \PY{n}{np}\PY{o}{.}\PY{n}{random}\PY{o}{.}\PY{n}{randn}\PY{p}{(}\PY{n}{n\PYZus{}MC}\PY{p}{,} \PY{n}{steps}\PY{p}{)}\PY{p}{)} \PY{c+c1}{\PYZsh{}\PYZsh{} logY=randn (1 ,Nt)}
    \PY{n}{Z} \PY{o}{=} \PY{n}{np}\PY{o}{.}\PY{n}{random}\PY{o}{.}\PY{n}{randn}\PY{p}{(}\PY{n}{n\PYZus{}MC}\PY{p}{,} \PY{n}{steps}\PY{p}{)}
    \PY{n}{X} \PY{o}{=} \PY{n}{np}\PY{o}{.}\PY{n}{zeros}\PY{p}{(}\PY{p}{(}\PY{n}{n\PYZus{}MC}\PY{p}{,} \PY{n}{steps}\PY{p}{)}\PY{p}{)}
    \PY{n}{X}\PY{p}{[}\PY{p}{:}\PY{p}{,} \PY{l+m+mi}{0}\PY{p}{]} \PY{o}{=} \PY{n}{np}\PY{o}{.}\PY{n}{log}\PY{p}{(}\PY{n}{S0}\PY{p}{)}
    \PY{k}{for} \PY{n}{t} \PY{o+ow}{in} \PY{n+nb}{range}\PY{p}{(}\PY{l+m+mi}{1}\PY{p}{,} \PY{n}{steps}\PY{p}{)}\PY{p}{:}
        \PY{n}{drift} \PY{o}{=} \PY{n}{r} \PY{o}{\PYZhy{}} \PY{p}{(}\PY{n}{lmbda} \PY{o}{*} \PY{p}{(}\PY{n}{np}\PY{o}{.}\PY{n}{exp}\PY{p}{(}\PY{n}{a} \PY{o}{+} \PY{l+m+mf}{0.5} \PY{o}{*} \PY{n}{b}\PY{o}{*}\PY{o}{*}\PY{l+m+mi}{2}\PY{p}{)} \PY{o}{\PYZhy{}} \PY{l+m+mi}{1}\PY{p}{)}\PY{p}{)} \PY{o}{\PYZhy{}} \PY{p}{(}\PY{l+m+mf}{0.5} \PY{o}{*} \PY{n}{sigma} \PY{o}{*}\PY{o}{*} \PY{l+m+mi}{2}\PY{p}{)}
        \PY{n}{diffusion} \PY{o}{=} \PY{n}{sigma} \PY{o}{*} \PY{n}{np}\PY{o}{.}\PY{n}{sqrt}\PY{p}{(}\PY{n}{dt}\PY{p}{)} \PY{o}{*} \PY{n}{Z}\PY{p}{[}\PY{p}{:}\PY{p}{,} \PY{n}{t}\PY{o}{\PYZhy{}}\PY{l+m+mi}{1}\PY{p}{]}
        \PY{n}{M} \PY{o}{=} \PY{n}{np}\PY{o}{.}\PY{n}{log}\PY{p}{(}\PY{n}{Y}\PY{p}{[}\PY{p}{:}\PY{p}{,} \PY{n}{t}\PY{o}{\PYZhy{}}\PY{l+m+mi}{1}\PY{p}{]}\PY{p}{)} \PY{o}{*} \PY{n}{Nt}\PY{p}{[}\PY{p}{:}\PY{p}{,} \PY{n}{t}\PY{o}{\PYZhy{}}\PY{l+m+mi}{1}\PY{p}{]}
        \PY{n}{X}\PY{p}{[}\PY{p}{:}\PY{p}{,} \PY{n}{t}\PY{p}{]} \PY{o}{=} \PY{n}{X}\PY{p}{[}\PY{p}{:}\PY{p}{,} \PY{n}{t}\PY{o}{\PYZhy{}}\PY{l+m+mi}{1}\PY{p}{]} \PY{o}{+} \PY{n}{drift} \PY{o}{*} \PY{n}{dt} \PY{o}{+} \PY{n}{diffusion} \PY{o}{+} \PY{n}{M}

    \PY{c+c1}{\PYZsh{} Compute option prices and implied volatilities}
    \PY{n}{S} \PY{o}{=} \PY{n}{np}\PY{o}{.}\PY{n}{exp}\PY{p}{(}\PY{n}{X}\PY{p}{)}
    \PY{k}{return} \PY{n}{S}\PY{p}{[}\PY{p}{:}\PY{p}{,}\PY{o}{\PYZhy{}}\PY{l+m+mi}{1}\PY{p}{]}
\end{Verbatim}
\end{tcolorbox}

    \hypertarget{function-2-calculate-call-option-price}{%
\paragraph{Function 2: Calculate Call Option
Price}\label{function-2-calculate-call-option-price}}

    \begin{tcolorbox}[breakable, size=fbox, boxrule=1pt, pad at break*=1mm,colback=cellbackground, colframe=cellborder]
\prompt{In}{incolor}{3}{\boxspacing}
\begin{Verbatim}[commandchars=\\\{\}]
\PY{k}{def} \PY{n+nf}{call\PYZus{}option\PYZus{}price}\PY{p}{(}\PY{n}{ST}\PY{p}{,} \PY{n}{Ks}\PY{p}{,} \PY{n}{r}\PY{p}{,} \PY{n}{T}\PY{p}{)}\PY{p}{:}
    \PY{n}{P} \PY{o}{=} \PY{n}{np}\PY{o}{.}\PY{n}{zeros}\PY{p}{(}\PY{p}{(}\PY{n+nb}{len}\PY{p}{(}\PY{n}{ST}\PY{p}{)}\PY{p}{,} \PY{n+nb}{len}\PY{p}{(}\PY{n}{Ks}\PY{p}{)}\PY{p}{)}\PY{p}{)}  \PY{c+c1}{\PYZsh{} Initialize array to hold option payoffs}
    \PY{n}{C} \PY{o}{=} \PY{n}{np}\PY{o}{.}\PY{n}{zeros}\PY{p}{(}\PY{n+nb}{len}\PY{p}{(}\PY{n}{Ks}\PY{p}{)}\PY{p}{)}             \PY{c+c1}{\PYZsh{} Initialize array to hold option prices}
    \PY{k}{for} \PY{n}{k}\PY{p}{,} \PY{n}{K} \PY{o+ow}{in} \PY{n+nb}{enumerate}\PY{p}{(}\PY{n}{Ks}\PY{p}{)}\PY{p}{:}
        \PY{c+c1}{\PYZsh{} Compute option payoffs for each stock price and strike price}
        \PY{n}{P}\PY{p}{[}\PY{p}{:}\PY{p}{,} \PY{n}{k}\PY{p}{]} \PY{o}{=} \PY{n}{np}\PY{o}{.}\PY{n}{maximum}\PY{p}{(}\PY{n}{ST}\PY{p}{[}\PY{p}{:}\PY{p}{]} \PY{o}{\PYZhy{}} \PY{n}{K}\PY{p}{,} \PY{l+m+mi}{0}\PY{p}{)}
        \PY{c+c1}{\PYZsh{} Compute the option price as the mean of the option payoffs, discounted to present value}
        \PY{n}{C}\PY{p}{[}\PY{n}{k}\PY{p}{]} \PY{o}{=} \PY{n}{np}\PY{o}{.}\PY{n}{mean}\PY{p}{(}\PY{n}{P}\PY{p}{[}\PY{p}{:}\PY{p}{,} \PY{n}{k}\PY{p}{]}\PY{p}{)} \PY{o}{*} \PY{n}{np}\PY{o}{.}\PY{n}{exp}\PY{p}{(}\PY{o}{\PYZhy{}}\PY{n}{r} \PY{o}{*} \PY{n}{T}\PY{p}{)}
    \PY{k}{return} \PY{n}{C}
\end{Verbatim}
\end{tcolorbox}

    \hypertarget{function-3-calculate-implied-volatility-using-newton-raphsons-method}{%
\paragraph{Function 3: Calculate Implied Volatility using Newton
Raphson's
Method}\label{function-3-calculate-implied-volatility-using-newton-raphsons-method}}

    \begin{tcolorbox}[breakable, size=fbox, boxrule=1pt, pad at break*=1mm,colback=cellbackground, colframe=cellborder]
\prompt{In}{incolor}{4}{\boxspacing}
\begin{Verbatim}[commandchars=\\\{\}]
\PY{k}{def} \PY{n+nf}{implied\PYZus{}volatility}\PY{p}{(}\PY{n}{S0}\PY{p}{,} \PY{n}{C}\PY{p}{,} \PY{n}{Ks}\PY{p}{,} \PY{n}{r}\PY{p}{,} \PY{n}{T}\PY{p}{,} \PY{n}{tol} \PY{o}{=} \PY{l+m+mf}{1e\PYZhy{}6}\PY{p}{,} \PY{n}{max\PYZus{}iter} \PY{o}{=} \PY{l+m+mi}{1000}\PY{p}{)}\PY{p}{:}
    \PY{n}{IV} \PY{o}{=} \PY{n}{np}\PY{o}{.}\PY{n}{zeros}\PY{p}{(}\PY{n+nb}{len}\PY{p}{(}\PY{n}{Ks}\PY{p}{)}\PY{p}{)}     \PY{c+c1}{\PYZsh{} Initialize array to hold implied volatilities}
    \PY{k}{for} \PY{n}{k}\PY{p}{,} \PY{n}{K} \PY{o+ow}{in} \PY{n+nb}{enumerate}\PY{p}{(}\PY{n}{Ks}\PY{p}{)}\PY{p}{:}
        \PY{c+c1}{\PYZsh{} Guess the implied volatility using the Black\PYZhy{}Scholes formula}
        \PY{n}{IV\PYZus{}guess} \PY{o}{=} \PY{n}{np}\PY{o}{.}\PY{n}{sqrt}\PY{p}{(}\PY{l+m+mi}{2} \PY{o}{*} \PY{n+nb}{abs}\PY{p}{(}\PY{p}{(}\PY{n}{np}\PY{o}{.}\PY{n}{log}\PY{p}{(}\PY{n}{S0}\PY{o}{/}\PY{n}{K}\PY{p}{)} \PY{o}{+} \PY{n}{r} \PY{o}{*} \PY{n}{T} \PY{o}{+} \PY{n}{C}\PY{p}{[}\PY{n}{k}\PY{p}{]}\PY{o}{/}\PY{n}{S0}\PY{p}{)} \PY{o}{/} \PY{n}{T}\PY{p}{)}\PY{p}{)}
        \PY{c+c1}{\PYZsh{} Iterate to refine the guess using Newton\PYZhy{}Raphson method}
        \PY{k}{for} \PY{n}{n} \PY{o+ow}{in} \PY{n+nb}{range}\PY{p}{(}\PY{n}{max\PYZus{}iter}\PY{p}{)}\PY{p}{:}
            \PY{c+c1}{\PYZsh{} Compute the option price and its sensitivity to volatility using Black\PYZhy{}Scholes formula}
            \PY{n}{d1} \PY{o}{=} \PY{p}{(}\PY{n}{np}\PY{o}{.}\PY{n}{log}\PY{p}{(}\PY{n}{S0}\PY{o}{/}\PY{n}{K}\PY{p}{)} \PY{o}{+} \PY{p}{(}\PY{n}{r} \PY{o}{+} \PY{n}{IV\PYZus{}guess}\PY{o}{*}\PY{o}{*}\PY{l+m+mi}{2}\PY{o}{/}\PY{l+m+mi}{2}\PY{p}{)} \PY{o}{*} \PY{n}{T}\PY{p}{)} \PY{o}{/} \PY{p}{(}\PY{n}{IV\PYZus{}guess} \PY{o}{*} \PY{n}{np}\PY{o}{.}\PY{n}{sqrt}\PY{p}{(}\PY{n}{T}\PY{p}{)}\PY{p}{)}
            \PY{n}{d2} \PY{o}{=} \PY{n}{d1} \PY{o}{\PYZhy{}} \PY{n}{IV\PYZus{}guess} \PY{o}{*} \PY{n}{np}\PY{o}{.}\PY{n}{sqrt}\PY{p}{(}\PY{n}{T}\PY{p}{)}
            \PY{n}{C\PYZus{}guess} \PY{o}{=} \PY{n}{S0} \PY{o}{*} \PY{n}{norm}\PY{o}{.}\PY{n}{cdf}\PY{p}{(}\PY{n}{d1}\PY{p}{)} \PY{o}{\PYZhy{}} \PY{n}{K} \PY{o}{*} \PY{n}{np}\PY{o}{.}\PY{n}{exp}\PY{p}{(}\PY{o}{\PYZhy{}}\PY{n}{r} \PY{o}{*} \PY{n}{T}\PY{p}{)} \PY{o}{*} \PY{n}{norm}\PY{o}{.}\PY{n}{cdf}\PY{p}{(}\PY{n}{d2}\PY{p}{)}
            \PY{n}{vega} \PY{o}{=} \PY{n}{S0} \PY{o}{*} \PY{n}{norm}\PY{o}{.}\PY{n}{pdf}\PY{p}{(}\PY{n}{d1}\PY{p}{)} \PY{o}{*} \PY{n}{np}\PY{o}{.}\PY{n}{sqrt}\PY{p}{(}\PY{n}{T}\PY{p}{)}
            \PY{c+c1}{\PYZsh{} Update the implied volatility using Newton\PYZhy{}Raphson formula}
            \PY{n}{IV\PYZus{}new} \PY{o}{=} \PY{n}{IV\PYZus{}guess} \PY{o}{\PYZhy{}} \PY{p}{(}\PY{n}{C\PYZus{}guess} \PY{o}{\PYZhy{}} \PY{n}{C}\PY{p}{[}\PY{n}{k}\PY{p}{]}\PY{p}{)} \PY{o}{/} \PY{p}{(}\PY{n}{vega} \PY{o}{+} \PY{l+m+mf}{1e\PYZhy{}4}\PY{p}{)}
            \PY{c+c1}{\PYZsh{} Check if the convergence criterion is satisfied}
            \PY{k}{if} \PY{n+nb}{abs}\PY{p}{(}\PY{n}{IV\PYZus{}new} \PY{o}{\PYZhy{}} \PY{n}{IV\PYZus{}guess}\PY{p}{)} \PY{o}{\PYZlt{}} \PY{n}{tol}\PY{p}{:}
                \PY{n}{IV}\PY{p}{[}\PY{n}{k}\PY{p}{]} \PY{o}{=} \PY{n}{IV\PYZus{}new}
                \PY{k}{break}
            \PY{k}{else}\PY{p}{:}
                \PY{n}{IV\PYZus{}guess} \PY{o}{=} \PY{n}{IV\PYZus{}new}
    \PY{k}{return} \PY{n}{IV}
\end{Verbatim}
\end{tcolorbox}

    \hypertarget{function-4-plot-each-graph-in-a-mn-subplot}{%
\paragraph{Function 4: Plot each graph in a m*n
subplot}\label{function-4-plot-each-graph-in-a-mn-subplot}}

    \begin{tcolorbox}[breakable, size=fbox, boxrule=1pt, pad at break*=1mm,colback=cellbackground, colframe=cellborder]
\prompt{In}{incolor}{5}{\boxspacing}
\begin{Verbatim}[commandchars=\\\{\}]
\PY{k}{def} \PY{n+nf}{plot\PYZus{}IV\PYZus{}vs\PYZus{}log\PYZus{}moneyness}\PY{p}{(}\PY{n}{i}\PY{p}{,} \PY{n}{j}\PY{p}{,} \PY{n}{lmbda}\PY{p}{,} \PY{n}{a}\PY{p}{,} \PY{n}{Ks}\PY{p}{,} \PY{n}{S0}\PY{p}{,} \PY{n}{IV}\PY{p}{,} \PY{n}{smooth}\PY{o}{=}\PY{k+kc}{False}\PY{p}{)}\PY{p}{:}
    \PY{l+s+sd}{\PYZdq{}\PYZdq{}\PYZdq{}}
\PY{l+s+sd}{    Plots implied volatility against log\PYZhy{}moneyness.}

\PY{l+s+sd}{    Args:}
\PY{l+s+sd}{    i (int): row index of the subplot}
\PY{l+s+sd}{    j (int): column index of the subplot}
\PY{l+s+sd}{    lmbda (float): parameter of the jump diffusion process}
\PY{l+s+sd}{    a (float): parameter of the jump diffusion process}
\PY{l+s+sd}{    Ks (ndarray): array of strike prices}
\PY{l+s+sd}{    S0 (float): initial stock price}
\PY{l+s+sd}{    IV (ndarray): array of implied volatility values}
\PY{l+s+sd}{    smooth (bool): whether to apply Savitzky\PYZhy{}Golay filter to smooth the line}

\PY{l+s+sd}{    Returns:}
\PY{l+s+sd}{    None}
\PY{l+s+sd}{    \PYZdq{}\PYZdq{}\PYZdq{}}
    \PY{k}{if} \PY{n}{smooth}\PY{p}{:}
        \PY{c+c1}{\PYZsh{} Apply Savitzky\PYZhy{}Golay filter to smoothen the line}
        \PY{n}{IV\PYZus{}smooth} \PY{o}{=} \PY{n}{savgol\PYZus{}filter}\PY{p}{(}\PY{n}{IV}\PY{p}{,} \PY{n}{window\PYZus{}length}\PY{o}{=}\PY{l+m+mi}{11}\PY{p}{,} \PY{n}{polyorder}\PY{o}{=}\PY{l+m+mi}{2}\PY{p}{)}
        \PY{n}{axs}\PY{p}{[}\PY{n}{i}\PY{p}{,}\PY{n}{j}\PY{p}{]}\PY{o}{.}\PY{n}{plot}\PY{p}{(}\PY{n}{np}\PY{o}{.}\PY{n}{log}\PY{p}{(}\PY{n}{Ks}\PY{o}{/}\PY{n}{S0}\PY{p}{)}\PY{p}{,} \PY{n}{IV\PYZus{}smooth}\PY{p}{,} \PY{l+s+s1}{\PYZsq{}}\PY{l+s+s1}{o\PYZhy{}}\PY{l+s+s1}{\PYZsq{}}\PY{p}{,} \PY{n}{label}\PY{o}{=}\PY{l+s+s1}{\PYZsq{}}\PY{l+s+s1}{Implied volatility}\PY{l+s+s1}{\PYZsq{}}\PY{p}{)}
    \PY{k}{else}\PY{p}{:}
        \PY{n}{axs}\PY{p}{[}\PY{n}{i}\PY{p}{,}\PY{n}{j}\PY{p}{]}\PY{o}{.}\PY{n}{plot}\PY{p}{(}\PY{n}{np}\PY{o}{.}\PY{n}{log}\PY{p}{(}\PY{n}{Ks}\PY{o}{/}\PY{n}{S0}\PY{p}{)}\PY{p}{,} \PY{n}{IV}\PY{p}{,} \PY{l+s+s1}{\PYZsq{}}\PY{l+s+s1}{o\PYZhy{}}\PY{l+s+s1}{\PYZsq{}}\PY{p}{)}
    \PY{n}{axs}\PY{p}{[}\PY{n}{i}\PY{p}{,}\PY{n}{j}\PY{p}{]}\PY{o}{.}\PY{n}{set\PYZus{}xlabel}\PY{p}{(}\PY{l+s+s1}{\PYZsq{}}\PY{l+s+s1}{Log\PYZhy{}moneyness}\PY{l+s+s1}{\PYZsq{}}\PY{p}{)}
    \PY{n}{axs}\PY{p}{[}\PY{n}{i}\PY{p}{,}\PY{n}{j}\PY{p}{]}\PY{o}{.}\PY{n}{set\PYZus{}ylabel}\PY{p}{(}\PY{l+s+s1}{\PYZsq{}}\PY{l+s+s1}{Implied volatility}\PY{l+s+s1}{\PYZsq{}}\PY{p}{)}
    \PY{n}{axs}\PY{p}{[}\PY{n}{i}\PY{p}{,}\PY{n}{j}\PY{p}{]}\PY{o}{.}\PY{n}{set\PYZus{}title}\PY{p}{(}\PY{l+s+sa}{f}\PY{l+s+s2}{\PYZdq{}}\PY{l+s+s2}{Lambda = }\PY{l+s+si}{\PYZob{}}\PY{n}{lmbda}\PY{l+s+si}{\PYZcb{}}\PY{l+s+s2}{, A = }\PY{l+s+si}{\PYZob{}}\PY{n}{a}\PY{l+s+si}{\PYZcb{}}\PY{l+s+s2}{\PYZdq{}}\PY{p}{)}
\end{Verbatim}
\end{tcolorbox}

    \hypertarget{set-model-parameters}{%
\subsubsection{Set Model Parameters}\label{set-model-parameters}}

    \begin{tcolorbox}[breakable, size=fbox, boxrule=1pt, pad at break*=1mm,colback=cellbackground, colframe=cellborder]
\prompt{In}{incolor}{6}{\boxspacing}
\begin{Verbatim}[commandchars=\\\{\}]
\PY{n}{S0} \PY{o}{=} \PY{l+m+mi}{100}
\PY{n}{r} \PY{o}{=} \PY{l+m+mf}{0.05}
\PY{n}{sigma} \PY{o}{=} \PY{l+m+mf}{0.3}
\PY{n}{T} \PY{o}{=} \PY{l+m+mi}{1}\PY{o}{/}\PY{l+m+mi}{12}
\PY{n}{dt} \PY{o}{=} \PY{l+m+mi}{1}\PY{o}{/}\PY{p}{(}\PY{l+m+mi}{4}\PY{o}{*}\PY{l+m+mi}{365}\PY{p}{)}
\PY{n}{b} \PY{o}{=} \PY{l+m+mf}{0.1}
\PY{n}{lmbda} \PY{o}{=} \PY{p}{[}\PY{l+m+mi}{2}\PY{p}{,} \PY{l+m+mi}{5}\PY{p}{]}
\PY{n}{a} \PY{o}{=} \PY{p}{[}\PY{o}{\PYZhy{}}\PY{l+m+mf}{0.1}\PY{p}{,} \PY{l+m+mf}{0.1}\PY{p}{]}
\end{Verbatim}
\end{tcolorbox}

    \hypertarget{set-option-parameters}{%
\subsubsection{Set option parameters}\label{set-option-parameters}}

    \begin{tcolorbox}[breakable, size=fbox, boxrule=1pt, pad at break*=1mm,colback=cellbackground, colframe=cellborder]
\prompt{In}{incolor}{7}{\boxspacing}
\begin{Verbatim}[commandchars=\\\{\}]
\PY{n}{Ks} \PY{o}{=} \PY{n}{np}\PY{o}{.}\PY{n}{arange}\PY{p}{(}\PY{l+m+mi}{90}\PY{p}{,} \PY{l+m+mi}{121}\PY{p}{)}
\PY{n}{n\PYZus{}Ks} \PY{o}{=} \PY{n+nb}{len}\PY{p}{(}\PY{n}{Ks}\PY{p}{)}
\PY{n}{n\PYZus{}MC} \PY{o}{=} \PY{l+m+mi}{20000}
\PY{n}{steps} \PY{o}{=} \PY{n+nb}{int}\PY{p}{(}\PY{n}{T}\PY{o}{/}\PY{n}{dt}\PY{p}{)} \PY{o}{+} \PY{l+m+mi}{1}
\end{Verbatim}
\end{tcolorbox}

    \hypertarget{initialize-arrays-to-store-results}{%
\subsubsection{Initialize arrays to store
results}\label{initialize-arrays-to-store-results}}

    \begin{tcolorbox}[breakable, size=fbox, boxrule=1pt, pad at break*=1mm,colback=cellbackground, colframe=cellborder]
\prompt{In}{incolor}{8}{\boxspacing}
\begin{Verbatim}[commandchars=\\\{\}]
\PY{n}{ST} \PY{o}{=} \PY{n}{np}\PY{o}{.}\PY{n}{zeros}\PY{p}{(}\PY{n}{n\PYZus{}MC}\PY{p}{)}
\end{Verbatim}
\end{tcolorbox}

    \hypertarget{create-a-mn-subplot}{%
\subsubsection{Create a m*n subplot}\label{create-a-mn-subplot}}

    \begin{tcolorbox}[breakable, size=fbox, boxrule=1pt, pad at break*=1mm,colback=cellbackground, colframe=cellborder]
\prompt{In}{incolor}{9}{\boxspacing}
\begin{Verbatim}[commandchars=\\\{\}]
\PY{c+c1}{\PYZsh{} create a figure with subplots}
\PY{n}{fig}\PY{p}{,} \PY{n}{axs} \PY{o}{=} \PY{n}{plt}\PY{o}{.}\PY{n}{subplots}\PY{p}{(}\PY{n}{nrows}\PY{o}{=}\PY{n+nb}{len}\PY{p}{(}\PY{n}{lmbda}\PY{p}{)}\PY{p}{,} \PY{n}{ncols}\PY{o}{=}\PY{n+nb}{len}\PY{p}{(}\PY{n}{a}\PY{p}{)}\PY{p}{,} \PY{n}{figsize}\PY{o}{=}\PY{p}{(}\PY{l+m+mi}{16}\PY{p}{,} \PY{l+m+mi}{10}\PY{p}{)}\PY{p}{)}

\PY{n}{start} \PY{o}{=} \PY{n}{time}\PY{p}{(}\PY{p}{)}

\PY{c+c1}{\PYZsh{} loop over each value of lambda and a to generate plots for each combination}
\PY{k}{for} \PY{n}{i}\PY{p}{,} \PY{n}{l} \PY{o+ow}{in} \PY{n+nb}{enumerate}\PY{p}{(}\PY{n}{lmbda}\PY{p}{)}\PY{p}{:}
    \PY{k}{for} \PY{n}{j}\PY{p}{,} \PY{n}{a0} \PY{o+ow}{in} \PY{n+nb}{enumerate}\PY{p}{(}\PY{n}{a}\PY{p}{)}\PY{p}{:}
        \PY{c+c1}{\PYZsh{} generate 20000 jump diffusion paths using given parameters and get prices at time T}
        \PY{n}{ST} \PY{o}{=} \PY{n}{jump\PYZus{}diffusion}\PY{p}{(}\PY{n}{S0}\PY{p}{,} \PY{n}{r}\PY{p}{,} \PY{n}{dt}\PY{p}{,} \PY{n}{sigma}\PY{p}{,} \PY{n}{steps}\PY{p}{,} \PY{n}{n\PYZus{}MC}\PY{p}{,} \PY{n}{l}\PY{p}{,} \PY{n}{a0}\PY{p}{,} \PY{n}{b}\PY{p}{)}

        \PY{c+c1}{\PYZsh{} calculate call option prices for each strike price using the generated jump diffusion paths}
        \PY{n}{C} \PY{o}{=} \PY{n}{call\PYZus{}option\PYZus{}price}\PY{p}{(}\PY{n}{ST}\PY{p}{,} \PY{n}{Ks}\PY{p}{,} \PY{n}{r}\PY{p}{,} \PY{n}{T}\PY{p}{)}

        \PY{c+c1}{\PYZsh{} calculate implied volatilities for each strike price using the call option prices}
        \PY{n}{IV} \PY{o}{=} \PY{n}{implied\PYZus{}volatility}\PY{p}{(}\PY{n}{S0}\PY{p}{,} \PY{n}{C}\PY{p}{,} \PY{n}{Ks}\PY{p}{,} \PY{n}{r}\PY{p}{,} \PY{n}{T}\PY{p}{,} \PY{n}{tol} \PY{o}{=} \PY{l+m+mf}{1e\PYZhy{}3}\PY{p}{,} \PY{n}{max\PYZus{}iter} \PY{o}{=} \PY{l+m+mi}{1000}\PY{p}{)}

        \PY{c+c1}{\PYZsh{} plot the implied volatility as a function of log\PYZhy{}moneyness using a Savitzky\PYZhy{}Golay filter to smoothen the line}
        \PY{n}{plot\PYZus{}IV\PYZus{}vs\PYZus{}log\PYZus{}moneyness}\PY{p}{(}\PY{n}{i}\PY{p}{,} \PY{n}{j}\PY{p}{,} \PY{n}{l}\PY{p}{,} \PY{n}{a0}\PY{p}{,} \PY{n}{Ks}\PY{p}{,} \PY{n}{S0}\PY{p}{,} \PY{n}{IV}\PY{p}{,} \PY{k+kc}{False}\PY{p}{)}
        
\PY{n}{end} \PY{o}{=} \PY{n}{time}\PY{p}{(}\PY{p}{)}

\PY{n+nb}{print}\PY{p}{(}\PY{l+s+s2}{\PYZdq{}}\PY{l+s+s2}{Total Computation time is:}\PY{l+s+s2}{\PYZdq{}}\PY{p}{,} \PY{n}{end} \PY{o}{\PYZhy{}} \PY{n}{start}\PY{p}{)}
\end{Verbatim}
\end{tcolorbox}

    \begin{Verbatim}[commandchars=\\\{\}]
Total Computation time is: 1.172874927520752
    \end{Verbatim}

    \begin{center}
    \adjustimage{max size={0.9\linewidth}{0.9\paperheight}}{output_19_1.png}
    \end{center}
    { \hspace*{\fill} \\}
    
    \hypertarget{conclusion}{%
\subsubsection{Conclusion:}\label{conclusion}}

\begin{itemize}
\item
  An increase in jump intensity parameter (lambda) causes the volatility
  smile to transition from a smile to a frown shape, leading to higher
  implied volatility. As lambda rises, the frequency of jumps increases,
  leading to a wider distribution of returns, which in turn contributes
  to higher implied volatility. The volatility frown may also reflect an
  increased preference for out-of-the-money options as investors seek to
  hedge against the potential risk of large jump movements.
\item
  On the other hand, an increase in the jump size parameter (a) causes
  the volatility smile to invert from a reverse skew to a forward skew
  shape. When a is large, investors may expect larger jumps, which may
  translate to higher implied volatility. The forward skew of the smile
  may indicate higher demand for out-of-the-money call options, which
  can act as a hedge against the risk of extreme upward jumps in the
  underlying asset.
\end{itemize}


    % Add a bibliography block to the postdoc
    
    
    
\end{document}
