\documentclass[11pt]{article}

    \usepackage[breakable]{tcolorbox}
    \usepackage{parskip} % Stop auto-indenting (to mimic markdown behaviour)
    

    % Basic figure setup, for now with no caption control since it's done
    % automatically by Pandoc (which extracts ![](path) syntax from Markdown).
    \usepackage{graphicx}
    % Maintain compatibility with old templates. Remove in nbconvert 6.0
    \let\Oldincludegraphics\includegraphics
    % Ensure that by default, figures have no caption (until we provide a
    % proper Figure object with a Caption API and a way to capture that
    % in the conversion process - todo).
    \usepackage{caption}
    \DeclareCaptionFormat{nocaption}{}
    \captionsetup{format=nocaption,aboveskip=0pt,belowskip=0pt}

    \usepackage{float}
    \floatplacement{figure}{H} % forces figures to be placed at the correct location
    \usepackage{xcolor} % Allow colors to be defined
    \usepackage{enumerate} % Needed for markdown enumerations to work
    \usepackage{geometry} % Used to adjust the document margins
    \usepackage{amsmath} % Equations
    \usepackage{amssymb} % Equations
    \usepackage{textcomp} % defines textquotesingle
    % Hack from http://tex.stackexchange.com/a/47451/13684:
    \AtBeginDocument{%
        \def\PYZsq{\textquotesingle}% Upright quotes in Pygmentized code
    }
    \usepackage{upquote} % Upright quotes for verbatim code
    \usepackage{eurosym} % defines \euro

    \usepackage{iftex}
    \ifPDFTeX
        \usepackage[T1]{fontenc}
        \IfFileExists{alphabeta.sty}{
              \usepackage{alphabeta}
          }{
              \usepackage[mathletters]{ucs}
              \usepackage[utf8x]{inputenc}
          }
    \else
        \usepackage{fontspec}
        \usepackage{unicode-math}
    \fi

    \usepackage{fancyvrb} % verbatim replacement that allows latex
    \usepackage{grffile} % extends the file name processing of package graphics 
                         % to support a larger range
    \makeatletter % fix for old versions of grffile with XeLaTeX
    \@ifpackagelater{grffile}{2019/11/01}
    {
      % Do nothing on new versions
    }
    {
      \def\Gread@@xetex#1{%
        \IfFileExists{"\Gin@base".bb}%
        {\Gread@eps{\Gin@base.bb}}%
        {\Gread@@xetex@aux#1}%
      }
    }
    \makeatother
    \usepackage[Export]{adjustbox} % Used to constrain images to a maximum size
    \adjustboxset{max size={0.9\linewidth}{0.9\paperheight}}

    % The hyperref package gives us a pdf with properly built
    % internal navigation ('pdf bookmarks' for the table of contents,
    % internal cross-reference links, web links for URLs, etc.)
    \usepackage{hyperref}
    % The default LaTeX title has an obnoxious amount of whitespace. By default,
    % titling removes some of it. It also provides customization options.
    \usepackage{titling}
    \usepackage{longtable} % longtable support required by pandoc >1.10
    \usepackage{booktabs}  % table support for pandoc > 1.12.2
    \usepackage{array}     % table support for pandoc >= 2.11.3
    \usepackage{calc}      % table minipage width calculation for pandoc >= 2.11.1
    \usepackage[inline]{enumitem} % IRkernel/repr support (it uses the enumerate* environment)
    \usepackage[normalem]{ulem} % ulem is needed to support strikethroughs (\sout)
                                % normalem makes italics be italics, not underlines
    \usepackage{mathrsfs}
    

    
    % Colors for the hyperref package
    \definecolor{urlcolor}{rgb}{0,.145,.698}
    \definecolor{linkcolor}{rgb}{.71,0.21,0.01}
    \definecolor{citecolor}{rgb}{.12,.54,.11}

    % ANSI colors
    \definecolor{ansi-black}{HTML}{3E424D}
    \definecolor{ansi-black-intense}{HTML}{282C36}
    \definecolor{ansi-red}{HTML}{E75C58}
    \definecolor{ansi-red-intense}{HTML}{B22B31}
    \definecolor{ansi-green}{HTML}{00A250}
    \definecolor{ansi-green-intense}{HTML}{007427}
    \definecolor{ansi-yellow}{HTML}{DDB62B}
    \definecolor{ansi-yellow-intense}{HTML}{B27D12}
    \definecolor{ansi-blue}{HTML}{208FFB}
    \definecolor{ansi-blue-intense}{HTML}{0065CA}
    \definecolor{ansi-magenta}{HTML}{D160C4}
    \definecolor{ansi-magenta-intense}{HTML}{A03196}
    \definecolor{ansi-cyan}{HTML}{60C6C8}
    \definecolor{ansi-cyan-intense}{HTML}{258F8F}
    \definecolor{ansi-white}{HTML}{C5C1B4}
    \definecolor{ansi-white-intense}{HTML}{A1A6B2}
    \definecolor{ansi-default-inverse-fg}{HTML}{FFFFFF}
    \definecolor{ansi-default-inverse-bg}{HTML}{000000}

    % common color for the border for error outputs.
    \definecolor{outerrorbackground}{HTML}{FFDFDF}

    % commands and environments needed by pandoc snippets
    % extracted from the output of `pandoc -s`
    \providecommand{\tightlist}{%
      \setlength{\itemsep}{0pt}\setlength{\parskip}{0pt}}
    \DefineVerbatimEnvironment{Highlighting}{Verbatim}{commandchars=\\\{\}}
    % Add ',fontsize=\small' for more characters per line
    \newenvironment{Shaded}{}{}
    \newcommand{\KeywordTok}[1]{\textcolor[rgb]{0.00,0.44,0.13}{\textbf{{#1}}}}
    \newcommand{\DataTypeTok}[1]{\textcolor[rgb]{0.56,0.13,0.00}{{#1}}}
    \newcommand{\DecValTok}[1]{\textcolor[rgb]{0.25,0.63,0.44}{{#1}}}
    \newcommand{\BaseNTok}[1]{\textcolor[rgb]{0.25,0.63,0.44}{{#1}}}
    \newcommand{\FloatTok}[1]{\textcolor[rgb]{0.25,0.63,0.44}{{#1}}}
    \newcommand{\CharTok}[1]{\textcolor[rgb]{0.25,0.44,0.63}{{#1}}}
    \newcommand{\StringTok}[1]{\textcolor[rgb]{0.25,0.44,0.63}{{#1}}}
    \newcommand{\CommentTok}[1]{\textcolor[rgb]{0.38,0.63,0.69}{\textit{{#1}}}}
    \newcommand{\OtherTok}[1]{\textcolor[rgb]{0.00,0.44,0.13}{{#1}}}
    \newcommand{\AlertTok}[1]{\textcolor[rgb]{1.00,0.00,0.00}{\textbf{{#1}}}}
    \newcommand{\FunctionTok}[1]{\textcolor[rgb]{0.02,0.16,0.49}{{#1}}}
    \newcommand{\RegionMarkerTok}[1]{{#1}}
    \newcommand{\ErrorTok}[1]{\textcolor[rgb]{1.00,0.00,0.00}{\textbf{{#1}}}}
    \newcommand{\NormalTok}[1]{{#1}}
    
    % Additional commands for more recent versions of Pandoc
    \newcommand{\ConstantTok}[1]{\textcolor[rgb]{0.53,0.00,0.00}{{#1}}}
    \newcommand{\SpecialCharTok}[1]{\textcolor[rgb]{0.25,0.44,0.63}{{#1}}}
    \newcommand{\VerbatimStringTok}[1]{\textcolor[rgb]{0.25,0.44,0.63}{{#1}}}
    \newcommand{\SpecialStringTok}[1]{\textcolor[rgb]{0.73,0.40,0.53}{{#1}}}
    \newcommand{\ImportTok}[1]{{#1}}
    \newcommand{\DocumentationTok}[1]{\textcolor[rgb]{0.73,0.13,0.13}{\textit{{#1}}}}
    \newcommand{\AnnotationTok}[1]{\textcolor[rgb]{0.38,0.63,0.69}{\textbf{\textit{{#1}}}}}
    \newcommand{\CommentVarTok}[1]{\textcolor[rgb]{0.38,0.63,0.69}{\textbf{\textit{{#1}}}}}
    \newcommand{\VariableTok}[1]{\textcolor[rgb]{0.10,0.09,0.49}{{#1}}}
    \newcommand{\ControlFlowTok}[1]{\textcolor[rgb]{0.00,0.44,0.13}{\textbf{{#1}}}}
    \newcommand{\OperatorTok}[1]{\textcolor[rgb]{0.40,0.40,0.40}{{#1}}}
    \newcommand{\BuiltInTok}[1]{{#1}}
    \newcommand{\ExtensionTok}[1]{{#1}}
    \newcommand{\PreprocessorTok}[1]{\textcolor[rgb]{0.74,0.48,0.00}{{#1}}}
    \newcommand{\AttributeTok}[1]{\textcolor[rgb]{0.49,0.56,0.16}{{#1}}}
    \newcommand{\InformationTok}[1]{\textcolor[rgb]{0.38,0.63,0.69}{\textbf{\textit{{#1}}}}}
    \newcommand{\WarningTok}[1]{\textcolor[rgb]{0.38,0.63,0.69}{\textbf{\textit{{#1}}}}}
    
    
    % Define a nice break command that doesn't care if a line doesn't already
    % exist.
    \def\br{\hspace*{\fill} \\* }
    % Math Jax compatibility definitions
    \def\gt{>}
    \def\lt{<}
    \let\Oldtex\TeX
    \let\Oldlatex\LaTeX
    \renewcommand{\TeX}{\textrm{\Oldtex}}
    \renewcommand{\LaTeX}{\textrm{\Oldlatex}}
    % Document parameters
    % Document title
    \title{Question1\_Asian\_Option\_Pricing}
    
    
    
    
    
% Pygments definitions
\makeatletter
\def\PY@reset{\let\PY@it=\relax \let\PY@bf=\relax%
    \let\PY@ul=\relax \let\PY@tc=\relax%
    \let\PY@bc=\relax \let\PY@ff=\relax}
\def\PY@tok#1{\csname PY@tok@#1\endcsname}
\def\PY@toks#1+{\ifx\relax#1\empty\else%
    \PY@tok{#1}\expandafter\PY@toks\fi}
\def\PY@do#1{\PY@bc{\PY@tc{\PY@ul{%
    \PY@it{\PY@bf{\PY@ff{#1}}}}}}}
\def\PY#1#2{\PY@reset\PY@toks#1+\relax+\PY@do{#2}}

\@namedef{PY@tok@w}{\def\PY@tc##1{\textcolor[rgb]{0.73,0.73,0.73}{##1}}}
\@namedef{PY@tok@c}{\let\PY@it=\textit\def\PY@tc##1{\textcolor[rgb]{0.24,0.48,0.48}{##1}}}
\@namedef{PY@tok@cp}{\def\PY@tc##1{\textcolor[rgb]{0.61,0.40,0.00}{##1}}}
\@namedef{PY@tok@k}{\let\PY@bf=\textbf\def\PY@tc##1{\textcolor[rgb]{0.00,0.50,0.00}{##1}}}
\@namedef{PY@tok@kp}{\def\PY@tc##1{\textcolor[rgb]{0.00,0.50,0.00}{##1}}}
\@namedef{PY@tok@kt}{\def\PY@tc##1{\textcolor[rgb]{0.69,0.00,0.25}{##1}}}
\@namedef{PY@tok@o}{\def\PY@tc##1{\textcolor[rgb]{0.40,0.40,0.40}{##1}}}
\@namedef{PY@tok@ow}{\let\PY@bf=\textbf\def\PY@tc##1{\textcolor[rgb]{0.67,0.13,1.00}{##1}}}
\@namedef{PY@tok@nb}{\def\PY@tc##1{\textcolor[rgb]{0.00,0.50,0.00}{##1}}}
\@namedef{PY@tok@nf}{\def\PY@tc##1{\textcolor[rgb]{0.00,0.00,1.00}{##1}}}
\@namedef{PY@tok@nc}{\let\PY@bf=\textbf\def\PY@tc##1{\textcolor[rgb]{0.00,0.00,1.00}{##1}}}
\@namedef{PY@tok@nn}{\let\PY@bf=\textbf\def\PY@tc##1{\textcolor[rgb]{0.00,0.00,1.00}{##1}}}
\@namedef{PY@tok@ne}{\let\PY@bf=\textbf\def\PY@tc##1{\textcolor[rgb]{0.80,0.25,0.22}{##1}}}
\@namedef{PY@tok@nv}{\def\PY@tc##1{\textcolor[rgb]{0.10,0.09,0.49}{##1}}}
\@namedef{PY@tok@no}{\def\PY@tc##1{\textcolor[rgb]{0.53,0.00,0.00}{##1}}}
\@namedef{PY@tok@nl}{\def\PY@tc##1{\textcolor[rgb]{0.46,0.46,0.00}{##1}}}
\@namedef{PY@tok@ni}{\let\PY@bf=\textbf\def\PY@tc##1{\textcolor[rgb]{0.44,0.44,0.44}{##1}}}
\@namedef{PY@tok@na}{\def\PY@tc##1{\textcolor[rgb]{0.41,0.47,0.13}{##1}}}
\@namedef{PY@tok@nt}{\let\PY@bf=\textbf\def\PY@tc##1{\textcolor[rgb]{0.00,0.50,0.00}{##1}}}
\@namedef{PY@tok@nd}{\def\PY@tc##1{\textcolor[rgb]{0.67,0.13,1.00}{##1}}}
\@namedef{PY@tok@s}{\def\PY@tc##1{\textcolor[rgb]{0.73,0.13,0.13}{##1}}}
\@namedef{PY@tok@sd}{\let\PY@it=\textit\def\PY@tc##1{\textcolor[rgb]{0.73,0.13,0.13}{##1}}}
\@namedef{PY@tok@si}{\let\PY@bf=\textbf\def\PY@tc##1{\textcolor[rgb]{0.64,0.35,0.47}{##1}}}
\@namedef{PY@tok@se}{\let\PY@bf=\textbf\def\PY@tc##1{\textcolor[rgb]{0.67,0.36,0.12}{##1}}}
\@namedef{PY@tok@sr}{\def\PY@tc##1{\textcolor[rgb]{0.64,0.35,0.47}{##1}}}
\@namedef{PY@tok@ss}{\def\PY@tc##1{\textcolor[rgb]{0.10,0.09,0.49}{##1}}}
\@namedef{PY@tok@sx}{\def\PY@tc##1{\textcolor[rgb]{0.00,0.50,0.00}{##1}}}
\@namedef{PY@tok@m}{\def\PY@tc##1{\textcolor[rgb]{0.40,0.40,0.40}{##1}}}
\@namedef{PY@tok@gh}{\let\PY@bf=\textbf\def\PY@tc##1{\textcolor[rgb]{0.00,0.00,0.50}{##1}}}
\@namedef{PY@tok@gu}{\let\PY@bf=\textbf\def\PY@tc##1{\textcolor[rgb]{0.50,0.00,0.50}{##1}}}
\@namedef{PY@tok@gd}{\def\PY@tc##1{\textcolor[rgb]{0.63,0.00,0.00}{##1}}}
\@namedef{PY@tok@gi}{\def\PY@tc##1{\textcolor[rgb]{0.00,0.52,0.00}{##1}}}
\@namedef{PY@tok@gr}{\def\PY@tc##1{\textcolor[rgb]{0.89,0.00,0.00}{##1}}}
\@namedef{PY@tok@ge}{\let\PY@it=\textit}
\@namedef{PY@tok@gs}{\let\PY@bf=\textbf}
\@namedef{PY@tok@gp}{\let\PY@bf=\textbf\def\PY@tc##1{\textcolor[rgb]{0.00,0.00,0.50}{##1}}}
\@namedef{PY@tok@go}{\def\PY@tc##1{\textcolor[rgb]{0.44,0.44,0.44}{##1}}}
\@namedef{PY@tok@gt}{\def\PY@tc##1{\textcolor[rgb]{0.00,0.27,0.87}{##1}}}
\@namedef{PY@tok@err}{\def\PY@bc##1{{\setlength{\fboxsep}{\string -\fboxrule}\fcolorbox[rgb]{1.00,0.00,0.00}{1,1,1}{\strut ##1}}}}
\@namedef{PY@tok@kc}{\let\PY@bf=\textbf\def\PY@tc##1{\textcolor[rgb]{0.00,0.50,0.00}{##1}}}
\@namedef{PY@tok@kd}{\let\PY@bf=\textbf\def\PY@tc##1{\textcolor[rgb]{0.00,0.50,0.00}{##1}}}
\@namedef{PY@tok@kn}{\let\PY@bf=\textbf\def\PY@tc##1{\textcolor[rgb]{0.00,0.50,0.00}{##1}}}
\@namedef{PY@tok@kr}{\let\PY@bf=\textbf\def\PY@tc##1{\textcolor[rgb]{0.00,0.50,0.00}{##1}}}
\@namedef{PY@tok@bp}{\def\PY@tc##1{\textcolor[rgb]{0.00,0.50,0.00}{##1}}}
\@namedef{PY@tok@fm}{\def\PY@tc##1{\textcolor[rgb]{0.00,0.00,1.00}{##1}}}
\@namedef{PY@tok@vc}{\def\PY@tc##1{\textcolor[rgb]{0.10,0.09,0.49}{##1}}}
\@namedef{PY@tok@vg}{\def\PY@tc##1{\textcolor[rgb]{0.10,0.09,0.49}{##1}}}
\@namedef{PY@tok@vi}{\def\PY@tc##1{\textcolor[rgb]{0.10,0.09,0.49}{##1}}}
\@namedef{PY@tok@vm}{\def\PY@tc##1{\textcolor[rgb]{0.10,0.09,0.49}{##1}}}
\@namedef{PY@tok@sa}{\def\PY@tc##1{\textcolor[rgb]{0.73,0.13,0.13}{##1}}}
\@namedef{PY@tok@sb}{\def\PY@tc##1{\textcolor[rgb]{0.73,0.13,0.13}{##1}}}
\@namedef{PY@tok@sc}{\def\PY@tc##1{\textcolor[rgb]{0.73,0.13,0.13}{##1}}}
\@namedef{PY@tok@dl}{\def\PY@tc##1{\textcolor[rgb]{0.73,0.13,0.13}{##1}}}
\@namedef{PY@tok@s2}{\def\PY@tc##1{\textcolor[rgb]{0.73,0.13,0.13}{##1}}}
\@namedef{PY@tok@sh}{\def\PY@tc##1{\textcolor[rgb]{0.73,0.13,0.13}{##1}}}
\@namedef{PY@tok@s1}{\def\PY@tc##1{\textcolor[rgb]{0.73,0.13,0.13}{##1}}}
\@namedef{PY@tok@mb}{\def\PY@tc##1{\textcolor[rgb]{0.40,0.40,0.40}{##1}}}
\@namedef{PY@tok@mf}{\def\PY@tc##1{\textcolor[rgb]{0.40,0.40,0.40}{##1}}}
\@namedef{PY@tok@mh}{\def\PY@tc##1{\textcolor[rgb]{0.40,0.40,0.40}{##1}}}
\@namedef{PY@tok@mi}{\def\PY@tc##1{\textcolor[rgb]{0.40,0.40,0.40}{##1}}}
\@namedef{PY@tok@il}{\def\PY@tc##1{\textcolor[rgb]{0.40,0.40,0.40}{##1}}}
\@namedef{PY@tok@mo}{\def\PY@tc##1{\textcolor[rgb]{0.40,0.40,0.40}{##1}}}
\@namedef{PY@tok@ch}{\let\PY@it=\textit\def\PY@tc##1{\textcolor[rgb]{0.24,0.48,0.48}{##1}}}
\@namedef{PY@tok@cm}{\let\PY@it=\textit\def\PY@tc##1{\textcolor[rgb]{0.24,0.48,0.48}{##1}}}
\@namedef{PY@tok@cpf}{\let\PY@it=\textit\def\PY@tc##1{\textcolor[rgb]{0.24,0.48,0.48}{##1}}}
\@namedef{PY@tok@c1}{\let\PY@it=\textit\def\PY@tc##1{\textcolor[rgb]{0.24,0.48,0.48}{##1}}}
\@namedef{PY@tok@cs}{\let\PY@it=\textit\def\PY@tc##1{\textcolor[rgb]{0.24,0.48,0.48}{##1}}}

\def\PYZbs{\char`\\}
\def\PYZus{\char`\_}
\def\PYZob{\char`\{}
\def\PYZcb{\char`\}}
\def\PYZca{\char`\^}
\def\PYZam{\char`\&}
\def\PYZlt{\char`\<}
\def\PYZgt{\char`\>}
\def\PYZsh{\char`\#}
\def\PYZpc{\char`\%}
\def\PYZdl{\char`\$}
\def\PYZhy{\char`\-}
\def\PYZsq{\char`\'}
\def\PYZdq{\char`\"}
\def\PYZti{\char`\~}
% for compatibility with earlier versions
\def\PYZat{@}
\def\PYZlb{[}
\def\PYZrb{]}
\makeatother


    % For linebreaks inside Verbatim environment from package fancyvrb. 
    \makeatletter
        \newbox\Wrappedcontinuationbox 
        \newbox\Wrappedvisiblespacebox 
        \newcommand*\Wrappedvisiblespace {\textcolor{red}{\textvisiblespace}} 
        \newcommand*\Wrappedcontinuationsymbol {\textcolor{red}{\llap{\tiny$\m@th\hookrightarrow$}}} 
        \newcommand*\Wrappedcontinuationindent {3ex } 
        \newcommand*\Wrappedafterbreak {\kern\Wrappedcontinuationindent\copy\Wrappedcontinuationbox} 
        % Take advantage of the already applied Pygments mark-up to insert 
        % potential linebreaks for TeX processing. 
        %        {, <, #, %, $, ' and ": go to next line. 
        %        _, }, ^, &, >, - and ~: stay at end of broken line. 
        % Use of \textquotesingle for straight quote. 
        \newcommand*\Wrappedbreaksatspecials {% 
            \def\PYGZus{\discretionary{\char`\_}{\Wrappedafterbreak}{\char`\_}}% 
            \def\PYGZob{\discretionary{}{\Wrappedafterbreak\char`\{}{\char`\{}}% 
            \def\PYGZcb{\discretionary{\char`\}}{\Wrappedafterbreak}{\char`\}}}% 
            \def\PYGZca{\discretionary{\char`\^}{\Wrappedafterbreak}{\char`\^}}% 
            \def\PYGZam{\discretionary{\char`\&}{\Wrappedafterbreak}{\char`\&}}% 
            \def\PYGZlt{\discretionary{}{\Wrappedafterbreak\char`\<}{\char`\<}}% 
            \def\PYGZgt{\discretionary{\char`\>}{\Wrappedafterbreak}{\char`\>}}% 
            \def\PYGZsh{\discretionary{}{\Wrappedafterbreak\char`\#}{\char`\#}}% 
            \def\PYGZpc{\discretionary{}{\Wrappedafterbreak\char`\%}{\char`\%}}% 
            \def\PYGZdl{\discretionary{}{\Wrappedafterbreak\char`\$}{\char`\$}}% 
            \def\PYGZhy{\discretionary{\char`\-}{\Wrappedafterbreak}{\char`\-}}% 
            \def\PYGZsq{\discretionary{}{\Wrappedafterbreak\textquotesingle}{\textquotesingle}}% 
            \def\PYGZdq{\discretionary{}{\Wrappedafterbreak\char`\"}{\char`\"}}% 
            \def\PYGZti{\discretionary{\char`\~}{\Wrappedafterbreak}{\char`\~}}% 
        } 
        % Some characters . , ; ? ! / are not pygmentized. 
        % This macro makes them "active" and they will insert potential linebreaks 
        \newcommand*\Wrappedbreaksatpunct {% 
            \lccode`\~`\.\lowercase{\def~}{\discretionary{\hbox{\char`\.}}{\Wrappedafterbreak}{\hbox{\char`\.}}}% 
            \lccode`\~`\,\lowercase{\def~}{\discretionary{\hbox{\char`\,}}{\Wrappedafterbreak}{\hbox{\char`\,}}}% 
            \lccode`\~`\;\lowercase{\def~}{\discretionary{\hbox{\char`\;}}{\Wrappedafterbreak}{\hbox{\char`\;}}}% 
            \lccode`\~`\:\lowercase{\def~}{\discretionary{\hbox{\char`\:}}{\Wrappedafterbreak}{\hbox{\char`\:}}}% 
            \lccode`\~`\?\lowercase{\def~}{\discretionary{\hbox{\char`\?}}{\Wrappedafterbreak}{\hbox{\char`\?}}}% 
            \lccode`\~`\!\lowercase{\def~}{\discretionary{\hbox{\char`\!}}{\Wrappedafterbreak}{\hbox{\char`\!}}}% 
            \lccode`\~`\/\lowercase{\def~}{\discretionary{\hbox{\char`\/}}{\Wrappedafterbreak}{\hbox{\char`\/}}}% 
            \catcode`\.\active
            \catcode`\,\active 
            \catcode`\;\active
            \catcode`\:\active
            \catcode`\?\active
            \catcode`\!\active
            \catcode`\/\active 
            \lccode`\~`\~ 	
        }
    \makeatother

    \let\OriginalVerbatim=\Verbatim
    \makeatletter
    \renewcommand{\Verbatim}[1][1]{%
        %\parskip\z@skip
        \sbox\Wrappedcontinuationbox {\Wrappedcontinuationsymbol}%
        \sbox\Wrappedvisiblespacebox {\FV@SetupFont\Wrappedvisiblespace}%
        \def\FancyVerbFormatLine ##1{\hsize\linewidth
            \vtop{\raggedright\hyphenpenalty\z@\exhyphenpenalty\z@
                \doublehyphendemerits\z@\finalhyphendemerits\z@
                \strut ##1\strut}%
        }%
        % If the linebreak is at a space, the latter will be displayed as visible
        % space at end of first line, and a continuation symbol starts next line.
        % Stretch/shrink are however usually zero for typewriter font.
        \def\FV@Space {%
            \nobreak\hskip\z@ plus\fontdimen3\font minus\fontdimen4\font
            \discretionary{\copy\Wrappedvisiblespacebox}{\Wrappedafterbreak}
            {\kern\fontdimen2\font}%
        }%
        
        % Allow breaks at special characters using \PYG... macros.
        \Wrappedbreaksatspecials
        % Breaks at punctuation characters . , ; ? ! and / need catcode=\active 	
        \OriginalVerbatim[#1,codes*=\Wrappedbreaksatpunct]%
    }
    \makeatother

    % Exact colors from NB
    \definecolor{incolor}{HTML}{303F9F}
    \definecolor{outcolor}{HTML}{D84315}
    \definecolor{cellborder}{HTML}{CFCFCF}
    \definecolor{cellbackground}{HTML}{F7F7F7}
    
    % prompt
    \makeatletter
    \newcommand{\boxspacing}{\kern\kvtcb@left@rule\kern\kvtcb@boxsep}
    \makeatother
    \newcommand{\prompt}[4]{
        {\ttfamily\llap{{\color{#2}[#3]:\hspace{3pt}#4}}\vspace{-\baselineskip}}
    }
    

    
    % Prevent overflowing lines due to hard-to-break entities
    \sloppy 
    % Setup hyperref package
    \hypersetup{
      breaklinks=true,  % so long urls are correctly broken across lines
      colorlinks=true,
      urlcolor=urlcolor,
      linkcolor=linkcolor,
      citecolor=citecolor,
      }
    % Slightly bigger margins than the latex defaults
    
    \geometry{verbose,tmargin=1in,bmargin=1in,lmargin=1in,rmargin=1in}
    
    

\begin{document}
    
    \maketitle
    
    

    
    \hypertarget{problem-1-pricing-an-asian-option}{%
\subsection{Problem 1: Pricing an Asian
Option}\label{problem-1-pricing-an-asian-option}}

    \hypertarget{import-libraries}{%
\subsubsection{Import Libraries}\label{import-libraries}}

    \begin{tcolorbox}[breakable, size=fbox, boxrule=1pt, pad at break*=1mm,colback=cellbackground, colframe=cellborder]
\prompt{In}{incolor}{1}{\boxspacing}
\begin{Verbatim}[commandchars=\\\{\}]
\PY{k+kn}{import} \PY{n+nn}{numpy} \PY{k}{as} \PY{n+nn}{np}
\PY{k+kn}{import} \PY{n+nn}{pandas} \PY{k}{as} \PY{n+nn}{pd}
\PY{k+kn}{import} \PY{n+nn}{math}
\PY{k+kn}{from} \PY{n+nn}{scipy}\PY{n+nn}{.}\PY{n+nn}{stats} \PY{k+kn}{import} \PY{n}{norm}
\PY{k+kn}{import} \PY{n+nn}{matplotlib}\PY{n+nn}{.}\PY{n+nn}{pyplot} \PY{k}{as} \PY{n+nn}{plt}
\PY{k+kn}{from} \PY{n+nn}{scipy}\PY{n+nn}{.}\PY{n+nn}{signal} \PY{k+kn}{import} \PY{n}{savgol\PYZus{}filter}
\end{Verbatim}
\end{tcolorbox}

    \hypertarget{set-parameters}{%
\subsubsection{Set Parameters}\label{set-parameters}}

    \begin{tcolorbox}[breakable, size=fbox, boxrule=1pt, pad at break*=1mm,colback=cellbackground, colframe=cellborder]
\prompt{In}{incolor}{2}{\boxspacing}
\begin{Verbatim}[commandchars=\\\{\}]
\PY{n}{T} \PY{o}{=} \PY{p}{[}\PY{l+m+mi}{1}\PY{p}{,}\PY{l+m+mi}{5}\PY{p}{]}
\PY{n}{m} \PY{o}{=} \PY{p}{[}\PY{l+m+mi}{4}\PY{p}{,}\PY{l+m+mi}{20}\PY{p}{]}
\PY{n}{S0} \PY{o}{=} \PY{l+m+mi}{100}
\PY{n}{r} \PY{o}{=} \PY{l+m+mf}{0.05}
\PY{n}{sigma} \PY{o}{=} \PY{l+m+mf}{0.2}
\PY{n}{N} \PY{o}{=} \PY{l+m+mi}{5000}
\PY{n}{Ks} \PY{o}{=} \PY{n}{np}\PY{o}{.}\PY{n}{arange}\PY{p}{(}\PY{l+m+mi}{90}\PY{p}{,} \PY{l+m+mi}{121}\PY{p}{)}
\end{Verbatim}
\end{tcolorbox}

    \hypertarget{define-functions}{%
\subsubsection{Define Functions}\label{define-functions}}

    \hypertarget{function-to-get-monte-carlo-simulation-price}{%
\paragraph{1. Function to get Monte Carlo Simulation
Price}\label{function-to-get-monte-carlo-simulation-price}}

    \begin{tcolorbox}[breakable, size=fbox, boxrule=1pt, pad at break*=1mm,colback=cellbackground, colframe=cellborder]
\prompt{In}{incolor}{3}{\boxspacing}
\begin{Verbatim}[commandchars=\\\{\}]
\PY{k}{def} \PY{n+nf}{monte\PYZus{}carlo\PYZus{}price}\PY{p}{(}\PY{n}{S0}\PY{p}{,} \PY{n}{K}\PY{p}{,} \PY{n}{r}\PY{p}{,} \PY{n}{sigma}\PY{p}{,} \PY{n}{T}\PY{p}{,} \PY{n}{m}\PY{p}{,} \PY{n}{N}\PY{p}{)}\PY{p}{:}
    \PY{l+s+sd}{\PYZdq{}\PYZdq{}\PYZdq{}}
\PY{l+s+sd}{    Computes the price of an Asian option using Monte Carlo simulation.}
\PY{l+s+sd}{    }
\PY{l+s+sd}{    Parameters:}
\PY{l+s+sd}{    S0 (float): initial stock price}
\PY{l+s+sd}{    K (float): strike price}
\PY{l+s+sd}{    r (float): risk\PYZhy{}free interest rate}
\PY{l+s+sd}{    sigma (float): volatility}
\PY{l+s+sd}{    T (float): time to maturity}
\PY{l+s+sd}{    m (int): number of time steps}
\PY{l+s+sd}{    N (int): number of simulations}
\PY{l+s+sd}{    }
\PY{l+s+sd}{    Returns:}
\PY{l+s+sd}{    float: the estimated price of the Asian Call option}
\PY{l+s+sd}{    \PYZdq{}\PYZdq{}\PYZdq{}}
    \PY{c+c1}{\PYZsh{} Compute time step size}
    \PY{n}{dt} \PY{o}{=} \PY{n}{T} \PY{o}{/} \PY{n}{m}
    
    \PY{c+c1}{\PYZsh{} Initialize array to store stock prices}
    \PY{n}{St} \PY{o}{=} \PY{n}{np}\PY{o}{.}\PY{n}{zeros}\PY{p}{(}\PY{p}{(}\PY{n}{N}\PY{p}{,} \PY{n}{m}\PY{o}{+}\PY{l+m+mi}{1}\PY{p}{)}\PY{p}{)}

    \PY{c+c1}{\PYZsh{} Set S0 as the starting point for simulation in each column}
    \PY{n}{St}\PY{p}{[}\PY{p}{:}\PY{p}{,} \PY{l+m+mi}{0}\PY{p}{]} \PY{o}{=} \PY{n}{S0} 
    
    \PY{c+c1}{\PYZsh{} Simulate stock price paths using standard Brownian motion}
    \PY{k}{for} \PY{n}{i} \PY{o+ow}{in} \PY{n+nb}{range}\PY{p}{(}\PY{l+m+mi}{1}\PY{p}{,} \PY{n}{m}\PY{o}{+}\PY{l+m+mi}{1}\PY{p}{)}\PY{p}{:}
        \PY{c+c1}{\PYZsh{} Generate N normally distributed random numbers}
        \PY{n}{dW} \PY{o}{=} \PY{n}{np}\PY{o}{.}\PY{n}{random}\PY{o}{.}\PY{n}{normal}\PY{p}{(}\PY{l+m+mi}{0}\PY{p}{,} \PY{n}{np}\PY{o}{.}\PY{n}{sqrt}\PY{p}{(}\PY{n}{dt}\PY{p}{)}\PY{p}{,} \PY{n}{size}\PY{o}{=}\PY{n}{N}\PY{p}{)}
        
        \PY{c+c1}{\PYZsh{} Update stock prices using the Black\PYZhy{}Scholes formula}
        \PY{n}{St}\PY{p}{[}\PY{p}{:}\PY{p}{,} \PY{n}{i}\PY{p}{]} \PY{o}{=} \PY{n}{St}\PY{p}{[}\PY{p}{:}\PY{p}{,} \PY{n}{i}\PY{o}{\PYZhy{}}\PY{l+m+mi}{1}\PY{p}{]} \PY{o}{*} \PY{n}{np}\PY{o}{.}\PY{n}{exp}\PY{p}{(}\PY{p}{(}\PY{n}{r} \PY{o}{\PYZhy{}} \PY{l+m+mf}{0.5} \PY{o}{*} \PY{n}{sigma}\PY{o}{*}\PY{o}{*}\PY{l+m+mi}{2}\PY{p}{)} \PY{o}{*} \PY{n}{dt} \PY{o}{+} \PY{n}{sigma} \PY{o}{*} \PY{n}{dW}\PY{p}{)}

    \PY{c+c1}{\PYZsh{} Compute the Asian option payoffs}
    \PY{n}{Sti} \PY{o}{=} \PY{n}{St}\PY{p}{[}\PY{p}{:}\PY{p}{,} \PY{l+m+mi}{1}\PY{p}{:}\PY{p}{]}
    \PY{n}{prices} \PY{o}{=} \PY{n}{np}\PY{o}{.}\PY{n}{exp}\PY{p}{(}\PY{o}{\PYZhy{}}\PY{n}{r} \PY{o}{*} \PY{n}{T}\PY{p}{)} \PY{o}{*} \PY{n}{np}\PY{o}{.}\PY{n}{maximum}\PY{p}{(}\PY{n}{np}\PY{o}{.}\PY{n}{sum}\PY{p}{(}\PY{n}{Sti}\PY{p}{,} \PY{n}{axis}\PY{o}{=}\PY{l+m+mi}{1}\PY{p}{)} \PY{o}{/} \PY{n}{m} \PY{o}{\PYZhy{}} \PY{n}{K}\PY{p}{,} \PY{l+m+mi}{0}\PY{p}{)}
    
    \PY{c+c1}{\PYZsh{} Save the average values from 5000 simulations in numpy array}
    \PY{n}{bach\PYZus{}dist} \PY{o}{=} \PY{n}{np}\PY{o}{.}\PY{n}{sum}\PY{p}{(}\PY{n}{Sti}\PY{p}{,} \PY{n}{axis}\PY{o}{=}\PY{l+m+mi}{1}\PY{p}{)} \PY{o}{/} \PY{n}{m}
    
    \PY{c+c1}{\PYZsh{} Compute the estimated price of the Asian option as the mean of the payoffs}
    \PY{n}{price} \PY{o}{=} \PY{n}{np}\PY{o}{.}\PY{n}{mean}\PY{p}{(}\PY{n}{prices}\PY{p}{)}
    
    \PY{c+c1}{\PYZsh{} Return Asian Call Option Price and simulation results to be used in Bachelier Model}
    \PY{k}{return} \PY{p}{[}\PY{n+nb}{round}\PY{p}{(}\PY{n}{price}\PY{p}{,} \PY{l+m+mi}{2}\PY{p}{)}\PY{p}{]}\PY{p}{,} \PY{n}{bach\PYZus{}dist}
\end{Verbatim}
\end{tcolorbox}

    \hypertarget{functions-to-get-price-through-log-normal-approximation-using-the-black-scholes-formula}{%
\paragraph{2. Functions to get price through log normal approximation
using the Black Scholes
Formula}\label{functions-to-get-price-through-log-normal-approximation-using-the-black-scholes-formula}}

    2(a). Black Scholes Formula (used in main function to calculate log
normal price)

    \begin{tcolorbox}[breakable, size=fbox, boxrule=1pt, pad at break*=1mm,colback=cellbackground, colframe=cellborder]
\prompt{In}{incolor}{4}{\boxspacing}
\begin{Verbatim}[commandchars=\\\{\}]
\PY{k}{def} \PY{n+nf}{black\PYZus{}scholes\PYZus{}formula}\PY{p}{(}\PY{n}{S0}\PY{p}{,} \PY{n}{sigma}\PY{p}{,} \PY{n}{r}\PY{p}{,} \PY{n}{T}\PY{p}{,} \PY{n}{K}\PY{p}{)}\PY{p}{:}
    \PY{l+s+sd}{\PYZdq{}\PYZdq{}\PYZdq{}}
\PY{l+s+sd}{    Computes the price of a European call option using the Black\PYZhy{}Scholes formula.}
\PY{l+s+sd}{    }
\PY{l+s+sd}{    Parameters:}
\PY{l+s+sd}{    S0 (float): initial stock price}
\PY{l+s+sd}{    sigma (float): volatility}
\PY{l+s+sd}{    r (float): risk\PYZhy{}free interest rate}
\PY{l+s+sd}{    T (float): time to maturity}
\PY{l+s+sd}{    K (float): strike price}
\PY{l+s+sd}{    }
\PY{l+s+sd}{    Returns:}
\PY{l+s+sd}{    float: the estimated price of the call option}
\PY{l+s+sd}{    \PYZdq{}\PYZdq{}\PYZdq{}}
    \PY{c+c1}{\PYZsh{} Compute d1 and d2, which are intermediate variables used in the Black\PYZhy{}Scholes formula}
    \PY{n}{d1} \PY{o}{=} \PY{p}{(}\PY{n}{math}\PY{o}{.}\PY{n}{log}\PY{p}{(}\PY{n}{S0}\PY{o}{/}\PY{n}{K}\PY{p}{)} \PY{o}{+} \PY{p}{(}\PY{n}{r} \PY{o}{+} \PY{p}{(}\PY{n}{sigma}\PY{o}{*}\PY{o}{*}\PY{l+m+mi}{2}\PY{o}{/}\PY{l+m+mi}{2}\PY{p}{)}\PY{p}{)}\PY{o}{*}\PY{p}{(}\PY{n}{T}\PY{p}{)}\PY{p}{)} \PY{o}{/} \PY{p}{(}\PY{n}{math}\PY{o}{.}\PY{n}{sqrt}\PY{p}{(}\PY{n}{T}\PY{p}{)}\PY{o}{*}\PY{n}{sigma}\PY{p}{)}
    \PY{n}{d2} \PY{o}{=} \PY{n}{d1} \PY{o}{\PYZhy{}} \PY{p}{(}\PY{n}{math}\PY{o}{.}\PY{n}{sqrt}\PY{p}{(}\PY{n}{T}\PY{p}{)} \PY{o}{*} \PY{n}{sigma}\PY{p}{)}
    
    \PY{c+c1}{\PYZsh{} Compute the price of the call option using the Black\PYZhy{}Scholes formula}
    \PY{n}{C} \PY{o}{=} \PY{p}{(}\PY{n}{S0} \PY{o}{*} \PY{n}{norm}\PY{o}{.}\PY{n}{cdf}\PY{p}{(}\PY{n}{d1}\PY{p}{)}\PY{p}{)} \PY{o}{\PYZhy{}} \PY{p}{(}\PY{n}{K} \PY{o}{*} \PY{n}{math}\PY{o}{.}\PY{n}{exp}\PY{p}{(}\PY{o}{\PYZhy{}}\PY{n}{r}\PY{o}{*}\PY{n}{T}\PY{p}{)} \PY{o}{*} \PY{n}{norm}\PY{o}{.}\PY{n}{cdf}\PY{p}{(}\PY{n}{d2}\PY{p}{)}\PY{p}{)}
    
    \PY{c+c1}{\PYZsh{} Round the estimated price to two decimal places and return it}
    \PY{k}{return} \PY{n+nb}{round}\PY{p}{(}\PY{n}{C}\PY{p}{,} \PY{l+m+mi}{2}\PY{p}{)}
\end{Verbatim}
\end{tcolorbox}

    2(b). Formula to calculate M2 used in log normal approximation pricing

    \begin{tcolorbox}[breakable, size=fbox, boxrule=1pt, pad at break*=1mm,colback=cellbackground, colframe=cellborder]
\prompt{In}{incolor}{5}{\boxspacing}
\begin{Verbatim}[commandchars=\\\{\}]
\PY{k}{def} \PY{n+nf}{calculate\PYZus{}M2}\PY{p}{(}\PY{n}{r}\PY{p}{,} \PY{n}{dt}\PY{p}{,} \PY{n}{sigma}\PY{p}{,} \PY{n}{S0}\PY{p}{,} \PY{n}{m}\PY{p}{)}\PY{p}{:}
    \PY{l+s+sd}{\PYZdq{}\PYZdq{}\PYZdq{}}
\PY{l+s+sd}{    Calculates the second moment of the stock price at maturity using the Euler\PYZhy{}Maruyama discretization.}
\PY{l+s+sd}{    }
\PY{l+s+sd}{    Parameters:}
\PY{l+s+sd}{    r (float): the risk\PYZhy{}free interest rate}
\PY{l+s+sd}{    dt (float): the time step size}
\PY{l+s+sd}{    sigma (float): the volatility of the stock price}
\PY{l+s+sd}{    S0 (float): the initial stock price}
\PY{l+s+sd}{    m (int): the number of time steps}
\PY{l+s+sd}{    }
\PY{l+s+sd}{    Returns:}
\PY{l+s+sd}{    float: the estimated second moment of the stock price at maturity}
\PY{l+s+sd}{    \PYZdq{}\PYZdq{}\PYZdq{}}
    \PY{n}{M2} \PY{o}{=} \PY{l+m+mi}{0}  \PY{c+c1}{\PYZsh{} Initialize the second moment}
    
    \PY{c+c1}{\PYZsh{} Compute the sum of the cross terms of the Euler\PYZhy{}Maruyama discretization}
    \PY{k}{for} \PY{n}{j} \PY{o+ow}{in} \PY{n+nb}{range}\PY{p}{(}\PY{l+m+mi}{1}\PY{p}{,} \PY{n}{m}\PY{o}{+}\PY{l+m+mi}{1}\PY{p}{)}\PY{p}{:}
        \PY{k}{for} \PY{n}{i} \PY{o+ow}{in} \PY{n+nb}{range}\PY{p}{(}\PY{l+m+mi}{1}\PY{p}{,} \PY{n}{j}\PY{p}{)}\PY{p}{:}
            \PY{n}{M2} \PY{o}{+}\PY{o}{=} \PY{l+m+mi}{2} \PY{o}{*} \PY{p}{(}\PY{p}{(}\PY{n}{S0} \PY{o}{*} \PY{n}{np}\PY{o}{.}\PY{n}{exp}\PY{p}{(}\PY{n}{r}\PY{o}{*}\PY{n}{i}\PY{o}{*}\PY{n}{dt}\PY{p}{)}\PY{p}{)} \PY{o}{*} \PY{p}{(}\PY{n}{S0} \PY{o}{*} \PY{n}{np}\PY{o}{.}\PY{n}{exp}\PY{p}{(}\PY{n}{r}\PY{o}{*}\PY{n}{j}\PY{o}{*}\PY{n}{dt}\PY{p}{)}\PY{p}{)} \PY{o}{*} \PY{p}{(}\PY{n}{np}\PY{o}{.}\PY{n}{exp}\PY{p}{(}\PY{p}{(}\PY{n}{sigma} \PY{o}{*}\PY{o}{*} \PY{l+m+mi}{2}\PY{p}{)} \PY{o}{*} \PY{n}{i} \PY{o}{*} \PY{n}{dt}\PY{p}{)}\PY{p}{)}\PY{p}{)}
    
    \PY{c+c1}{\PYZsh{} Compute the sum of the squared terms of the Euler\PYZhy{}Maruyama discretization}
    \PY{k}{for} \PY{n}{i} \PY{o+ow}{in} \PY{n+nb}{range}\PY{p}{(}\PY{l+m+mi}{1}\PY{p}{,} \PY{n}{m}\PY{o}{+}\PY{l+m+mi}{1}\PY{p}{)}\PY{p}{:}
        \PY{n}{M2} \PY{o}{+}\PY{o}{=} \PY{p}{(}\PY{p}{(}\PY{n}{S0} \PY{o}{*} \PY{n}{np}\PY{o}{.}\PY{n}{exp}\PY{p}{(}\PY{n}{r}\PY{o}{*}\PY{n}{i}\PY{o}{*}\PY{n}{dt}\PY{p}{)}\PY{p}{)}\PY{o}{*}\PY{o}{*}\PY{l+m+mi}{2}\PY{p}{)} \PY{o}{*} \PY{n}{np}\PY{o}{.}\PY{n}{exp}\PY{p}{(}\PY{p}{(}\PY{n}{sigma}\PY{o}{*}\PY{o}{*}\PY{l+m+mi}{2}\PY{p}{)}\PY{o}{*}\PY{p}{(}\PY{n}{i}\PY{o}{*}\PY{n}{dt}\PY{p}{)}\PY{p}{)}
    
    \PY{c+c1}{\PYZsh{} Compute the estimated second moment by dividing by the square of the number of time steps}
    \PY{k}{return} \PY{p}{(}\PY{n}{M2} \PY{o}{/} \PY{p}{(}\PY{n}{m}\PY{o}{*}\PY{o}{*}\PY{l+m+mi}{2}\PY{p}{)}\PY{p}{)}
\end{Verbatim}
\end{tcolorbox}

    2(c). Main function to calculate log normal price

    \begin{tcolorbox}[breakable, size=fbox, boxrule=1pt, pad at break*=1mm,colback=cellbackground, colframe=cellborder]
\prompt{In}{incolor}{6}{\boxspacing}
\begin{Verbatim}[commandchars=\\\{\}]
\PY{k}{def} \PY{n+nf}{log\PYZus{}normal\PYZus{}price}\PY{p}{(}\PY{n}{S0}\PY{p}{,} \PY{n}{K}\PY{p}{,} \PY{n}{r}\PY{p}{,} \PY{n}{sigma}\PY{p}{,} \PY{n}{T}\PY{p}{,} \PY{n}{m}\PY{p}{)}\PY{p}{:}
    \PY{l+s+sd}{\PYZdq{}\PYZdq{}\PYZdq{}}
\PY{l+s+sd}{    Computes the price of an Asian option using the log\PYZhy{}normal approximation method.}
\PY{l+s+sd}{    }
\PY{l+s+sd}{    Parameters:}
\PY{l+s+sd}{    S0 (float): the initial stock price}
\PY{l+s+sd}{    K (float): the strike price}
\PY{l+s+sd}{    r (float): the risk\PYZhy{}free interest rate}
\PY{l+s+sd}{    sigma (float): the volatility of the stock price}
\PY{l+s+sd}{    T (float): the time to maturity of the option}
\PY{l+s+sd}{    m (int): the number of time steps}
\PY{l+s+sd}{    }
\PY{l+s+sd}{    Returns:}
\PY{l+s+sd}{    float: the estimated price of the Asian option using the log\PYZhy{}normal approximation method}
\PY{l+s+sd}{    \PYZdq{}\PYZdq{}\PYZdq{}}
    \PY{c+c1}{\PYZsh{} Compute time step size}
    \PY{n}{dt} \PY{o}{=} \PY{n}{T}\PY{o}{/}\PY{n}{m}
    
    \PY{c+c1}{\PYZsh{} Compute the first moment of the stock price at maturity}
    \PY{n}{M1} \PY{o}{=} \PY{n}{np}\PY{o}{.}\PY{n}{sum}\PY{p}{(}\PY{n}{np}\PY{o}{.}\PY{n}{array}\PY{p}{(}\PY{p}{[}\PY{n}{S0}\PY{o}{*}\PY{n}{np}\PY{o}{.}\PY{n}{exp}\PY{p}{(}\PY{n}{r}\PY{o}{*}\PY{n}{i}\PY{o}{*}\PY{n}{dt}\PY{p}{)} \PY{k}{for} \PY{n}{i} \PY{o+ow}{in} \PY{n+nb}{range}\PY{p}{(}\PY{l+m+mi}{1}\PY{p}{,} \PY{n}{m}\PY{o}{+}\PY{l+m+mi}{1}\PY{p}{)}\PY{p}{]}\PY{p}{)}\PY{p}{)} \PY{o}{/} \PY{n}{m}
    
    \PY{c+c1}{\PYZsh{} Compute the second moment of the stock price at maturity by calling the calculate\PYZus{}M2 function defined above}
    \PY{n}{M2} \PY{o}{=} \PY{n}{calculate\PYZus{}M2}\PY{p}{(}\PY{n}{r}\PY{p}{,} \PY{n}{dt}\PY{p}{,} \PY{n}{sigma}\PY{p}{,} \PY{n}{S0}\PY{p}{,} \PY{n}{m}\PY{p}{)}
    
    \PY{c+c1}{\PYZsh{} Compute the estimated volatility and drift of the stock price using the log\PYZhy{}normal approximation method}
    \PY{n}{sigma\PYZus{}hat} \PY{o}{=} \PY{n}{math}\PY{o}{.}\PY{n}{sqrt}\PY{p}{(}\PY{p}{(}\PY{l+m+mi}{1}\PY{o}{/}\PY{n}{T}\PY{p}{)} \PY{o}{*} \PY{n}{math}\PY{o}{.}\PY{n}{log}\PY{p}{(}\PY{n}{M2} \PY{o}{/} \PY{p}{(}\PY{n}{M1}\PY{o}{*}\PY{o}{*}\PY{l+m+mi}{2}\PY{p}{)}\PY{p}{)}\PY{p}{)}
    \PY{n}{S0\PYZus{}hat} \PY{o}{=} \PY{n}{M1} \PY{o}{*} \PY{n}{np}\PY{o}{.}\PY{n}{exp}\PY{p}{(}\PY{o}{\PYZhy{}}\PY{n}{r} \PY{o}{*} \PY{n}{T}\PY{p}{)}
    
    \PY{c+c1}{\PYZsh{} Use the estimated volatility and drift to compute the price of the Asian option using the Black\PYZhy{}Scholes formula}
    \PY{k}{return} \PY{n}{black\PYZus{}scholes\PYZus{}formula}\PY{p}{(}\PY{n}{S0\PYZus{}hat}\PY{p}{,} \PY{n}{sigma\PYZus{}hat}\PY{p}{,} \PY{n}{r}\PY{p}{,} \PY{n}{T}\PY{p}{,} \PY{n}{K}\PY{p}{)}
\end{Verbatim}
\end{tcolorbox}

    \hypertarget{function-to-get-prices-with-normal-approximation-using-the-bachelier-call-formula}{%
\paragraph{3. Function to get prices with normal approximation using the
Bachelier Call
Formula}\label{function-to-get-prices-with-normal-approximation-using-the-bachelier-call-formula}}

    \begin{tcolorbox}[breakable, size=fbox, boxrule=1pt, pad at break*=1mm,colback=cellbackground, colframe=cellborder]
\prompt{In}{incolor}{7}{\boxspacing}
\begin{Verbatim}[commandchars=\\\{\}]
\PY{k}{def} \PY{n+nf}{bachelier\PYZus{}price}\PY{p}{(}\PY{n}{bach\PYZus{}dist}\PY{p}{,} \PY{n}{K}\PY{p}{,} \PY{n}{r}\PY{p}{,} \PY{n}{T}\PY{p}{)}\PY{p}{:}
    \PY{l+s+sd}{\PYZdq{}\PYZdq{}\PYZdq{}}
\PY{l+s+sd}{    Computes the price of an Asian option using the Bachelier Normal approximation method.}
\PY{l+s+sd}{    }
\PY{l+s+sd}{    Parameters:}
\PY{l+s+sd}{    S0 (float): the initial stock price}
\PY{l+s+sd}{    K (float): the strike price}
\PY{l+s+sd}{    r (float): the risk\PYZhy{}free interest rate}
\PY{l+s+sd}{    sigma (float): the volatility of the stock price}
\PY{l+s+sd}{    T (float): the time to maturity of the option}
\PY{l+s+sd}{    }
\PY{l+s+sd}{    Returns:}
\PY{l+s+sd}{    float: the estimated price of the Asian option using the Normal approximation method}
\PY{l+s+sd}{    \PYZdq{}\PYZdq{}\PYZdq{}}

    \PY{c+c1}{\PYZsh{} Calculate F and sigma for bachelier distribution}
    \PY{n}{Fb} \PY{o}{=} \PY{n}{bach\PYZus{}dist}\PY{o}{.}\PY{n}{mean}\PY{p}{(}\PY{p}{)}
    \PY{n}{sigmaB} \PY{o}{=} \PY{n}{bach\PYZus{}dist}\PY{o}{.}\PY{n}{std}\PY{p}{(}\PY{p}{)}
    
    \PY{c+c1}{\PYZsh{} Calculate the standardized normal variable}
    \PY{n}{Z} \PY{o}{=} \PY{p}{(}\PY{n}{Fb} \PY{o}{\PYZhy{}} \PY{n}{K}\PY{p}{)} \PY{o}{/} \PY{p}{(}\PY{n}{sigmaB}\PY{p}{)}
    
    \PY{c+c1}{\PYZsh{} Use the Bachelier Call formula to calculate the option price}
    \PY{n}{C} \PY{o}{=} \PY{n}{np}\PY{o}{.}\PY{n}{exp}\PY{p}{(}\PY{o}{\PYZhy{}}\PY{n}{r}\PY{o}{*}\PY{n}{T}\PY{p}{)} \PY{o}{*} \PY{p}{(}\PY{p}{(}\PY{n}{Fb} \PY{o}{\PYZhy{}} \PY{n}{K}\PY{p}{)}\PY{o}{*}\PY{n}{norm}\PY{o}{.}\PY{n}{cdf}\PY{p}{(}\PY{n}{Z}\PY{p}{)} \PY{o}{+} \PY{p}{(}\PY{n}{sigmaB} \PY{o}{*} \PY{n}{norm}\PY{o}{.}\PY{n}{pdf}\PY{p}{(}\PY{n}{Z}\PY{p}{,} \PY{n}{loc}\PY{o}{=}\PY{l+m+mi}{0}\PY{p}{,} \PY{n}{scale}\PY{o}{=}\PY{l+m+mi}{1}\PY{p}{)}\PY{p}{)}\PY{p}{)}
    
    \PY{c+c1}{\PYZsh{} Round the option price to two decimal places and return}
    \PY{k}{return} \PY{n+nb}{round}\PY{p}{(}\PY{n}{C}\PY{p}{,} \PY{l+m+mi}{2}\PY{p}{)}
\end{Verbatim}
\end{tcolorbox}

    \hypertarget{main-code}{%
\subsubsection{Main Code:}\label{main-code}}

\begin{verbatim}
- Fetches the asian option price through three methods defined
- Runs the code of a range of strike prices and for different combinations of (T, m)
- Stores the results in a dataframe and prints it
\end{verbatim}

    \begin{tcolorbox}[breakable, size=fbox, boxrule=1pt, pad at break*=1mm,colback=cellbackground, colframe=cellborder]
\prompt{In}{incolor}{8}{\boxspacing}
\begin{Verbatim}[commandchars=\\\{\}]
\PY{c+c1}{\PYZsh{} Initialize two empty dictionaries to store Asian option call prices for two different values of (T,m) i.e. (1,4) and (5,20)}
\PY{n}{asian\PYZus{}option\PYZus{}1\PYZus{}4} \PY{o}{=} \PY{p}{\PYZob{}}\PY{p}{\PYZcb{}}
\PY{n}{asian\PYZus{}option\PYZus{}5\PYZus{}20} \PY{o}{=} \PY{p}{\PYZob{}}\PY{p}{\PYZcb{}}

\PY{c+c1}{\PYZsh{} For each strike price (K), fetch the call option prices using}
\PY{c+c1}{\PYZsh{} Monte Carlo, log normal, and Bachelier pricing models for the two different (T,m) values}

\PY{k}{for} \PY{n}{K} \PY{o+ow}{in} \PY{n}{Ks}\PY{p}{:}
    \PY{c+c1}{\PYZsh{} For (1,4) calculate the call option prices and store in the dictionary asian\PYZus{}option\PYZus{}1\PYZus{}4}
    \PY{n}{asian\PYZus{}option\PYZus{}1\PYZus{}4}\PY{p}{[}\PY{n}{K}\PY{p}{]}\PY{p}{,} \PY{n}{bach\PYZus{}dist\PYZus{}1} \PY{o}{=} \PY{n}{monte\PYZus{}carlo\PYZus{}price}\PY{p}{(}\PY{n}{S0}\PY{p}{,} \PY{n}{K}\PY{p}{,} \PY{n}{r}\PY{p}{,} \PY{n}{sigma}\PY{p}{,} \PY{n}{T}\PY{p}{[}\PY{l+m+mi}{0}\PY{p}{]}\PY{p}{,} \PY{n}{m}\PY{p}{[}\PY{l+m+mi}{0}\PY{p}{]}\PY{p}{,} \PY{n}{N}\PY{p}{)}
    \PY{n}{asian\PYZus{}option\PYZus{}1\PYZus{}4}\PY{p}{[}\PY{n}{K}\PY{p}{]}\PY{o}{.}\PY{n}{append}\PY{p}{(}\PY{n}{log\PYZus{}normal\PYZus{}price}\PY{p}{(}\PY{n}{S0}\PY{p}{,} \PY{n}{K}\PY{p}{,} \PY{n}{r}\PY{p}{,} \PY{n}{sigma}\PY{p}{,} \PY{n}{T}\PY{p}{[}\PY{l+m+mi}{0}\PY{p}{]}\PY{p}{,} \PY{n}{m}\PY{p}{[}\PY{l+m+mi}{0}\PY{p}{]}\PY{p}{)}\PY{p}{)}
    \PY{n}{asian\PYZus{}option\PYZus{}1\PYZus{}4}\PY{p}{[}\PY{n}{K}\PY{p}{]}\PY{o}{.}\PY{n}{append}\PY{p}{(}\PY{n}{bachelier\PYZus{}price}\PY{p}{(}\PY{n}{bach\PYZus{}dist\PYZus{}1}\PY{p}{,} \PY{n}{K}\PY{p}{,} \PY{n}{r}\PY{p}{,} \PY{n}{T}\PY{p}{[}\PY{l+m+mi}{0}\PY{p}{]}\PY{p}{)}\PY{p}{)}
    
    \PY{c+c1}{\PYZsh{} For (5,20) calculate the call option prices and store in the dictionary asian\PYZus{}option\PYZus{}5\PYZus{}20}
    \PY{n}{asian\PYZus{}option\PYZus{}5\PYZus{}20}\PY{p}{[}\PY{n}{K}\PY{p}{]}\PY{p}{,} \PY{n}{bach\PYZus{}dist\PYZus{}2} \PY{o}{=} \PY{n}{monte\PYZus{}carlo\PYZus{}price}\PY{p}{(}\PY{n}{S0}\PY{p}{,} \PY{n}{K}\PY{p}{,} \PY{n}{r}\PY{p}{,} \PY{n}{sigma}\PY{p}{,} \PY{n}{T}\PY{p}{[}\PY{l+m+mi}{1}\PY{p}{]}\PY{p}{,} \PY{n}{m}\PY{p}{[}\PY{l+m+mi}{1}\PY{p}{]}\PY{p}{,} \PY{n}{N}\PY{p}{)}
    \PY{n}{asian\PYZus{}option\PYZus{}5\PYZus{}20}\PY{p}{[}\PY{n}{K}\PY{p}{]}\PY{o}{.}\PY{n}{append}\PY{p}{(}\PY{n}{log\PYZus{}normal\PYZus{}price}\PY{p}{(}\PY{n}{S0}\PY{p}{,} \PY{n}{K}\PY{p}{,} \PY{n}{r}\PY{p}{,} \PY{n}{sigma}\PY{p}{,} \PY{n}{T}\PY{p}{[}\PY{l+m+mi}{1}\PY{p}{]}\PY{p}{,} \PY{n}{m}\PY{p}{[}\PY{l+m+mi}{1}\PY{p}{]}\PY{p}{)}\PY{p}{)}
    \PY{n}{asian\PYZus{}option\PYZus{}5\PYZus{}20}\PY{p}{[}\PY{n}{K}\PY{p}{]}\PY{o}{.}\PY{n}{append}\PY{p}{(}\PY{n}{bachelier\PYZus{}price}\PY{p}{(}\PY{n}{bach\PYZus{}dist\PYZus{}2}\PY{p}{,} \PY{n}{K}\PY{p}{,} \PY{n}{r}\PY{p}{,} \PY{n}{T}\PY{p}{[}\PY{l+m+mi}{1}\PY{p}{]}\PY{p}{)}\PY{p}{)}

\PY{c+c1}{\PYZsh{} Convert the dictionaries asian\PYZus{}option\PYZus{}1\PYZus{}4 and asian\PYZus{}option\PYZus{}5\PYZus{}20 into pandas DataFrames with the strike prices as the index and the prices obtained from each pricing model as columns}
\PY{n}{asian\PYZus{}price\PYZus{}1\PYZus{}4\PYZus{}df} \PY{o}{=} \PY{n}{pd}\PY{o}{.}\PY{n}{DataFrame}\PY{o}{.}\PY{n}{from\PYZus{}dict}\PY{p}{(}\PY{n}{asian\PYZus{}option\PYZus{}1\PYZus{}4}\PY{p}{,} \PY{n}{orient}\PY{o}{=}\PY{l+s+s1}{\PYZsq{}}\PY{l+s+s1}{index}\PY{l+s+s1}{\PYZsq{}}\PY{p}{,} \PY{n}{columns}\PY{o}{=}\PY{p}{[}\PY{l+s+s1}{\PYZsq{}}\PY{l+s+s1}{Monte Carlo}\PY{l+s+s1}{\PYZsq{}}\PY{p}{,} \PY{l+s+s1}{\PYZsq{}}\PY{l+s+s1}{log normal}\PY{l+s+s1}{\PYZsq{}}\PY{p}{,}\PY{l+s+s1}{\PYZsq{}}\PY{l+s+s1}{Bachelier}\PY{l+s+s1}{\PYZsq{}}\PY{p}{]}\PY{p}{)}
\PY{n}{asian\PYZus{}price\PYZus{}5\PYZus{}20\PYZus{}df} \PY{o}{=} \PY{n}{pd}\PY{o}{.}\PY{n}{DataFrame}\PY{o}{.}\PY{n}{from\PYZus{}dict}\PY{p}{(}\PY{n}{asian\PYZus{}option\PYZus{}5\PYZus{}20}\PY{p}{,} \PY{n}{orient}\PY{o}{=}\PY{l+s+s1}{\PYZsq{}}\PY{l+s+s1}{index}\PY{l+s+s1}{\PYZsq{}}\PY{p}{,} \PY{n}{columns}\PY{o}{=}\PY{p}{[}\PY{l+s+s1}{\PYZsq{}}\PY{l+s+s1}{Monte Carlo}\PY{l+s+s1}{\PYZsq{}}\PY{p}{,} \PY{l+s+s1}{\PYZsq{}}\PY{l+s+s1}{log normal}\PY{l+s+s1}{\PYZsq{}}\PY{p}{,}\PY{l+s+s1}{\PYZsq{}}\PY{l+s+s1}{Bachelier}\PY{l+s+s1}{\PYZsq{}}\PY{p}{]}\PY{p}{)}

\PY{c+c1}{\PYZsh{} Rename the index of the DataFrames to \PYZsq{}Strike\PYZsq{}}
\PY{n}{asian\PYZus{}price\PYZus{}1\PYZus{}4\PYZus{}df} \PY{o}{=} \PY{n}{asian\PYZus{}price\PYZus{}1\PYZus{}4\PYZus{}df}\PY{o}{.}\PY{n}{rename\PYZus{}axis}\PY{p}{(}\PY{l+s+s1}{\PYZsq{}}\PY{l+s+s1}{Strike}\PY{l+s+s1}{\PYZsq{}}\PY{p}{)}
\PY{n}{asian\PYZus{}price\PYZus{}5\PYZus{}20\PYZus{}df} \PY{o}{=} \PY{n}{asian\PYZus{}price\PYZus{}5\PYZus{}20\PYZus{}df}\PY{o}{.}\PY{n}{rename\PYZus{}axis}\PY{p}{(}\PY{l+s+s1}{\PYZsq{}}\PY{l+s+s1}{Strike}\PY{l+s+s1}{\PYZsq{}}\PY{p}{)}
\end{Verbatim}
\end{tcolorbox}

    \begin{tcolorbox}[breakable, size=fbox, boxrule=1pt, pad at break*=1mm,colback=cellbackground, colframe=cellborder]
\prompt{In}{incolor}{9}{\boxspacing}
\begin{Verbatim}[commandchars=\\\{\}]
\PY{c+c1}{\PYZsh{} Display asian option prices for So = \PYZdl{}100, r = 5\PYZpc{}, sigma = 0.2, T = 1 and m = 4}
\PY{n+nb}{print}\PY{p}{(}\PY{l+s+s2}{\PYZdq{}}\PY{l+s+s2}{Asian Option Price when T = 1 and m = 4:}\PY{l+s+se}{\PYZbs{}n}\PY{l+s+se}{\PYZbs{}n}\PY{l+s+s2}{\PYZdq{}}\PY{p}{,} \PY{n}{asian\PYZus{}price\PYZus{}1\PYZus{}4\PYZus{}df}\PY{p}{)}
\end{Verbatim}
\end{tcolorbox}

    \begin{Verbatim}[commandchars=\\\{\}]
Asian Option Price when T = 1 and m = 4:

         Monte Carlo  log normal  Bachelier
Strike
90            13.62       13.61      13.91
91            12.88       12.83      13.19
92            12.07       12.08      12.30
93            11.15       11.35      11.40
94            10.75       10.65      11.01
95             9.82        9.97      10.02
96             9.24        9.31       9.46
97             8.60        8.68       8.84
98             8.26        8.08       8.44
99             7.52        7.50       7.67
100            6.90        6.95       7.01
101            6.25        6.43       6.34
102            6.09        5.94       6.16
103            5.35        5.47       5.37
104            4.92        5.03       4.94
105            4.56        4.61       4.55
106            4.03        4.22       3.99
107            3.88        3.86       3.78
108            3.51        3.52       3.38
109            3.04        3.20       2.93
110            2.97        2.91       2.86
111            2.72        2.64       2.57
112            2.44        2.39       2.29
113            2.29        2.16       2.12
114            1.91        1.95       1.73
115            1.73        1.75       1.55
116            1.44        1.57       1.22
117            1.46        1.41       1.23
118            1.22        1.26       1.00
119            1.07        1.13       0.87
120            1.04        1.00       0.81
    \end{Verbatim}

    \begin{tcolorbox}[breakable, size=fbox, boxrule=1pt, pad at break*=1mm,colback=cellbackground, colframe=cellborder]
\prompt{In}{incolor}{10}{\boxspacing}
\begin{Verbatim}[commandchars=\\\{\}]
\PY{c+c1}{\PYZsh{} Display asian option prices for So = \PYZdl{}100, r = 5\PYZpc{}, sigma = 0.2, T = 5 and m = 20}
\PY{n+nb}{print}\PY{p}{(}\PY{l+s+s2}{\PYZdq{}}\PY{l+s+s2}{Asian Option Price when T = 5 and m = 20:}\PY{l+s+se}{\PYZbs{}n}\PY{l+s+se}{\PYZbs{}n}\PY{l+s+s2}{\PYZdq{}}\PY{p}{,} \PY{n}{asian\PYZus{}price\PYZus{}5\PYZus{}20\PYZus{}df}\PY{p}{)}
\end{Verbatim}
\end{tcolorbox}

    \begin{Verbatim}[commandchars=\\\{\}]
Asian Option Price when T = 5 and m = 20:

         Monte Carlo  log normal  Bachelier
Strike
90            21.50       21.29      22.58
91            20.31       20.70      21.32
92            20.23       20.12      21.39
93            18.44       19.55      19.48
94            19.05       18.99      20.06
95            18.23       18.44      19.36
96            18.34       17.90      19.46
97            17.29       17.37      18.34
98            16.95       16.84      17.93
99            15.83       16.34      16.82
100           15.48       15.84      16.40
101           16.05       15.35      16.97
102           14.82       14.87      15.66
103           14.08       14.40      14.80
104           13.84       13.94      14.72
105           13.80       13.50      14.67
106           12.57       13.06      13.35
107           12.49       12.64      13.03
108           11.64       12.22      12.17
109           12.28       11.82      12.97
110           11.52       11.42      11.92
111           10.98       11.04      11.38
112           10.19       10.66      10.55
113            9.96       10.30      10.35
114            9.69        9.95       9.91
115           10.02        9.60      10.32
116            9.40        9.27       9.64
117            8.48        8.94       8.59
118            8.05        8.62       8.18
119            8.31        8.32       8.45
120            8.19        8.02       8.32
    \end{Verbatim}

    \hypertarget{conclusion}{%
\subsubsection{Conclusion:}\label{conclusion}}

At T = 1, all three prices almost overlap and give similar results.
Based on the analysis conducted, it can be concluded that for shorter
periods, both the log-normal and Bachelier approximation methods provide
accurate results, with the log-normal method performing slightly better.
However, as the time to maturity increases, the Bachelier method begins
to produce prices with larger errors, while the log-normal method
continues to provide good estimates. This is evident from the fact that,
at T = 5 years, the log-normal prices closely match the Monte Carlo
estimates, while the Bachelier prices exhibit relatively poor results.
Hence, for longer time periods, the log-normal method is more reliable
in accurately estimating prices.

    \hypertarget{additional-analysis-by-plotting}{%
\subsubsection{Additional Analysis by
plotting}\label{additional-analysis-by-plotting}}

    \begin{tcolorbox}[breakable, size=fbox, boxrule=1pt, pad at break*=1mm,colback=cellbackground, colframe=cellborder]
\prompt{In}{incolor}{11}{\boxspacing}
\begin{Verbatim}[commandchars=\\\{\}]
\PY{l+s+sd}{\PYZdq{}\PYZdq{}\PYZdq{} }
\PY{l+s+sd}{Plot Asian Option Price obtained through }
\PY{l+s+sd}{Monte Carlo Simulation (n = 5000), Log Normal Approximation and Bachelier Normal Approximation}
\PY{l+s+sd}{when T = 1 and m = 4}
\PY{l+s+sd}{\PYZdq{}\PYZdq{}\PYZdq{}}
\PY{c+c1}{\PYZsh{} Apply Savitzky\PYZhy{}Golay filter to smoothen the line}
\PY{n}{asian\PYZus{}price\PYZus{}1\PYZus{}4\PYZus{}df}\PY{p}{[}\PY{l+s+s1}{\PYZsq{}}\PY{l+s+s1}{Monte Carlo Smooth}\PY{l+s+s1}{\PYZsq{}}\PY{p}{]} \PY{o}{=} \PY{n}{savgol\PYZus{}filter}\PY{p}{(}\PY{n}{asian\PYZus{}price\PYZus{}1\PYZus{}4\PYZus{}df}\PY{p}{[}\PY{l+s+s1}{\PYZsq{}}\PY{l+s+s1}{Monte Carlo}\PY{l+s+s1}{\PYZsq{}}\PY{p}{]}\PY{p}{,} \PY{n}{window\PYZus{}length}\PY{o}{=}\PY{l+m+mi}{21}\PY{p}{,} \PY{n}{polyorder}\PY{o}{=}\PY{l+m+mi}{2}\PY{p}{)}
\PY{n}{asian\PYZus{}price\PYZus{}1\PYZus{}4\PYZus{}df}\PY{p}{[}\PY{l+s+s1}{\PYZsq{}}\PY{l+s+s1}{Bachelier Smooth}\PY{l+s+s1}{\PYZsq{}}\PY{p}{]} \PY{o}{=} \PY{n}{savgol\PYZus{}filter}\PY{p}{(}\PY{n}{asian\PYZus{}price\PYZus{}1\PYZus{}4\PYZus{}df}\PY{p}{[}\PY{l+s+s1}{\PYZsq{}}\PY{l+s+s1}{Bachelier}\PY{l+s+s1}{\PYZsq{}}\PY{p}{]}\PY{p}{,} \PY{n}{window\PYZus{}length}\PY{o}{=}\PY{l+m+mi}{21}\PY{p}{,} \PY{n}{polyorder}\PY{o}{=}\PY{l+m+mi}{2}\PY{p}{)}

\PY{c+c1}{\PYZsh{} Define the figure and axes objects}
\PY{n}{fig}\PY{p}{,} \PY{n}{ax} \PY{o}{=} \PY{n}{plt}\PY{o}{.}\PY{n}{subplots}\PY{p}{(}\PY{n}{figsize}\PY{o}{=}\PY{p}{(}\PY{l+m+mi}{10}\PY{p}{,} \PY{l+m+mi}{6}\PY{p}{)}\PY{p}{)}

\PY{c+c1}{\PYZsh{} Plot the three columns against the index}
\PY{n}{asian\PYZus{}price\PYZus{}1\PYZus{}4\PYZus{}df}\PY{o}{.}\PY{n}{plot}\PY{p}{(}\PY{n}{ax}\PY{o}{=}\PY{n}{ax}\PY{p}{,} \PY{n}{y}\PY{o}{=}\PY{p}{[}\PY{l+s+s1}{\PYZsq{}}\PY{l+s+s1}{Monte Carlo Smooth}\PY{l+s+s1}{\PYZsq{}}\PY{p}{,} \PY{l+s+s1}{\PYZsq{}}\PY{l+s+s1}{log normal}\PY{l+s+s1}{\PYZsq{}}\PY{p}{,} \PY{l+s+s1}{\PYZsq{}}\PY{l+s+s1}{Bachelier Smooth}\PY{l+s+s1}{\PYZsq{}}\PY{p}{]}\PY{p}{,} \PY{n}{linestyle}\PY{o}{=}\PY{p}{(}\PY{l+s+s1}{\PYZsq{}}\PY{l+s+s1}{\PYZhy{}.}\PY{l+s+s1}{\PYZsq{}}\PY{p}{)}\PY{p}{)}

\PY{c+c1}{\PYZsh{} Set the labels and title}
\PY{n}{ax}\PY{o}{.}\PY{n}{set\PYZus{}xlabel}\PY{p}{(}\PY{l+s+s1}{\PYZsq{}}\PY{l+s+s1}{Strike Price}\PY{l+s+s1}{\PYZsq{}}\PY{p}{)}
\PY{n}{ax}\PY{o}{.}\PY{n}{set\PYZus{}ylabel}\PY{p}{(}\PY{l+s+s1}{\PYZsq{}}\PY{l+s+s1}{Asian Option Price}\PY{l+s+s1}{\PYZsq{}}\PY{p}{)}
\PY{n}{ax}\PY{o}{.}\PY{n}{set\PYZus{}title}\PY{p}{(}\PY{l+s+s1}{\PYZsq{}}\PY{l+s+s1}{Asian Option Price for So = \PYZdl{}100, r = 5}\PY{l+s+s1}{\PYZpc{}}\PY{l+s+s1}{, sigma = 0.2, T = 1 and m = 4}\PY{l+s+s1}{\PYZsq{}}\PY{p}{)}

\PY{c+c1}{\PYZsh{} Add the legend}
\PY{n}{ax}\PY{o}{.}\PY{n}{legend}\PY{p}{(}\PY{p}{[}\PY{l+s+s1}{\PYZsq{}}\PY{l+s+s1}{Monte Carlo (n=5000)}\PY{l+s+s1}{\PYZsq{}}\PY{p}{,} \PY{l+s+s1}{\PYZsq{}}\PY{l+s+s1}{Log Normal}\PY{l+s+s1}{\PYZsq{}}\PY{p}{,} \PY{l+s+s1}{\PYZsq{}}\PY{l+s+s1}{Bachelier}\PY{l+s+s1}{\PYZsq{}}\PY{p}{]}\PY{p}{)}

\PY{c+c1}{\PYZsh{} Show the plot}
\PY{n}{plt}\PY{o}{.}\PY{n}{show}\PY{p}{(}\PY{p}{)}
\end{Verbatim}
\end{tcolorbox}

    \begin{center}
    \adjustimage{max size={0.9\linewidth}{0.9\paperheight}}{output_23_0.png}
    \end{center}
    { \hspace*{\fill} \\}
    
    \begin{tcolorbox}[breakable, size=fbox, boxrule=1pt, pad at break*=1mm,colback=cellbackground, colframe=cellborder]
\prompt{In}{incolor}{12}{\boxspacing}
\begin{Verbatim}[commandchars=\\\{\}]
\PY{l+s+sd}{\PYZdq{}\PYZdq{}\PYZdq{} }
\PY{l+s+sd}{Plot Asian Option Price obtained through }
\PY{l+s+sd}{Monte Carlo Simulation (n = 5000), Log Normal Approximation and Bachelier Normal Approximation}
\PY{l+s+sd}{when T = 5 and m = 20}
\PY{l+s+sd}{\PYZdq{}\PYZdq{}\PYZdq{}}
\PY{c+c1}{\PYZsh{} Apply Savitzky\PYZhy{}Golay filter to smoothen the line}
\PY{n}{asian\PYZus{}price\PYZus{}5\PYZus{}20\PYZus{}df}\PY{p}{[}\PY{l+s+s1}{\PYZsq{}}\PY{l+s+s1}{Monte Carlo Smooth}\PY{l+s+s1}{\PYZsq{}}\PY{p}{]} \PY{o}{=} \PY{n}{savgol\PYZus{}filter}\PY{p}{(}\PY{n}{asian\PYZus{}price\PYZus{}5\PYZus{}20\PYZus{}df}\PY{p}{[}\PY{l+s+s1}{\PYZsq{}}\PY{l+s+s1}{Monte Carlo}\PY{l+s+s1}{\PYZsq{}}\PY{p}{]}\PY{p}{,} \PY{n}{window\PYZus{}length}\PY{o}{=}\PY{l+m+mi}{21}\PY{p}{,} \PY{n}{polyorder}\PY{o}{=}\PY{l+m+mi}{2}\PY{p}{)}
\PY{n}{asian\PYZus{}price\PYZus{}5\PYZus{}20\PYZus{}df}\PY{p}{[}\PY{l+s+s1}{\PYZsq{}}\PY{l+s+s1}{Bachelier Smooth}\PY{l+s+s1}{\PYZsq{}}\PY{p}{]} \PY{o}{=} \PY{n}{savgol\PYZus{}filter}\PY{p}{(}\PY{n}{asian\PYZus{}price\PYZus{}5\PYZus{}20\PYZus{}df}\PY{p}{[}\PY{l+s+s1}{\PYZsq{}}\PY{l+s+s1}{Bachelier}\PY{l+s+s1}{\PYZsq{}}\PY{p}{]}\PY{p}{,} \PY{n}{window\PYZus{}length}\PY{o}{=}\PY{l+m+mi}{21}\PY{p}{,} \PY{n}{polyorder}\PY{o}{=}\PY{l+m+mi}{2}\PY{p}{)}

\PY{c+c1}{\PYZsh{} Define the figure and axes objects}
\PY{n}{fig}\PY{p}{,} \PY{n}{ax} \PY{o}{=} \PY{n}{plt}\PY{o}{.}\PY{n}{subplots}\PY{p}{(}\PY{n}{figsize}\PY{o}{=}\PY{p}{(}\PY{l+m+mi}{10}\PY{p}{,} \PY{l+m+mi}{6}\PY{p}{)}\PY{p}{)}

\PY{c+c1}{\PYZsh{} Plot the three columns against the index}
\PY{n}{asian\PYZus{}price\PYZus{}5\PYZus{}20\PYZus{}df}\PY{o}{.}\PY{n}{plot}\PY{p}{(}\PY{n}{ax}\PY{o}{=}\PY{n}{ax}\PY{p}{,} \PY{n}{y}\PY{o}{=}\PY{p}{[}\PY{l+s+s1}{\PYZsq{}}\PY{l+s+s1}{Monte Carlo Smooth}\PY{l+s+s1}{\PYZsq{}}\PY{p}{,} \PY{l+s+s1}{\PYZsq{}}\PY{l+s+s1}{log normal}\PY{l+s+s1}{\PYZsq{}}\PY{p}{,} \PY{l+s+s1}{\PYZsq{}}\PY{l+s+s1}{Bachelier Smooth}\PY{l+s+s1}{\PYZsq{}}\PY{p}{]}\PY{p}{,} \PY{n}{linestyle}\PY{o}{=}\PY{p}{(}\PY{l+s+s1}{\PYZsq{}}\PY{l+s+s1}{\PYZhy{}.}\PY{l+s+s1}{\PYZsq{}}\PY{p}{)}\PY{p}{)}

\PY{c+c1}{\PYZsh{} Set the labels and title}
\PY{n}{ax}\PY{o}{.}\PY{n}{set\PYZus{}xlabel}\PY{p}{(}\PY{l+s+s1}{\PYZsq{}}\PY{l+s+s1}{Strike Price}\PY{l+s+s1}{\PYZsq{}}\PY{p}{)}
\PY{n}{ax}\PY{o}{.}\PY{n}{set\PYZus{}ylabel}\PY{p}{(}\PY{l+s+s1}{\PYZsq{}}\PY{l+s+s1}{Asian Option Price}\PY{l+s+s1}{\PYZsq{}}\PY{p}{)}
\PY{n}{ax}\PY{o}{.}\PY{n}{set\PYZus{}title}\PY{p}{(}\PY{l+s+s1}{\PYZsq{}}\PY{l+s+s1}{Asian Option Price for So = \PYZdl{}100, r = 5}\PY{l+s+s1}{\PYZpc{}}\PY{l+s+s1}{, sigma = 0.2, T = 5 and m = 20}\PY{l+s+s1}{\PYZsq{}}\PY{p}{)}

\PY{c+c1}{\PYZsh{} Add the legend}
\PY{n}{ax}\PY{o}{.}\PY{n}{legend}\PY{p}{(}\PY{p}{[}\PY{l+s+s1}{\PYZsq{}}\PY{l+s+s1}{Monte Carlo (n=5000)}\PY{l+s+s1}{\PYZsq{}}\PY{p}{,} \PY{l+s+s1}{\PYZsq{}}\PY{l+s+s1}{Log Normal}\PY{l+s+s1}{\PYZsq{}}\PY{p}{,} \PY{l+s+s1}{\PYZsq{}}\PY{l+s+s1}{Bachelier}\PY{l+s+s1}{\PYZsq{}}\PY{p}{]}\PY{p}{)}

\PY{c+c1}{\PYZsh{} Show the plot}
\PY{n}{plt}\PY{o}{.}\PY{n}{show}\PY{p}{(}\PY{p}{)}
\end{Verbatim}
\end{tcolorbox}

    \begin{center}
    \adjustimage{max size={0.9\linewidth}{0.9\paperheight}}{output_24_0.png}
    \end{center}
    { \hspace*{\fill} \\}
    

    % Add a bibliography block to the postdoc
    
    
    
\end{document}
